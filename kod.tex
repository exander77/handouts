\documentclass[justified]{tufte-book}
\setlength{\parindent}{0em}
\usepackage[english]{babel}
\usepackage[utf8x]{inputenc}
\usepackage[T1]{fontenc}

\hypersetup{colorlinks}% uncomment this line if you prefer colored hyperlinks (e.g., for onscreen viewing)

%\usepackage{indentfirst}
\usepackage{amsmath}
\usepackage{amsthm}
\usepackage{afterpage}
\usepackage{tikz}
\usepackage{forest}
\usetikzlibrary{tikzmark}

\newcommand{\floor}[1]{\lfloor #1 \rfloor}
\newcommand{\ceil}[1]{\lceil #1 \rceil}

\newtheorem{dt}{Definition}

%%
% Book metadata
\title[Data Compression]{
  %\setlength{\parindent}{0pt}
  Data Compression \\
  \thanks{Thanks.}
}
\author[Ing. Radom\'ir Pol\'ach et al.]{
  %\setlength{\parindent}{0pt}
  \\
  Ing. Radom\'ir Pol\'ach, \\ 
  prof. Ing. Jan Holub, Ph.D.\\
}
%\publisher{Publisher of This Book}


%%
% If they're installed, use Bergamo and Chantilly from www.fontsite.com.
% They're clones of Bembo and Gill Sans, respectively.
%\IfFileExists{bergamo.sty}{\usepackage[osf]{bergamo}}{}% Bembo
%\IfFileExists{chantill.sty}{\usepackage{chantill}}{}% Gill Sans

%\usepackage{microtype}

%%
% Just some sample text
\usepackage{lipsum}

%%
% For nicely typeset tabular material
\usepackage{booktabs}

%%
% For graphics / images
\usepackage{graphicx}
\setkeys{Gin}{width=\linewidth,totalheight=\textheight,keepaspectratio}
\graphicspath{{graphics/}}

% The fancyvrb package lets us customize the formatting of verbatim
% environments.  We use a slightly smaller font.
\usepackage{fancyvrb}
\fvset{fontsize=\normalsize}

%%
% Prints argument within hanging parentheses (i.e., parentheses that take
% up no horizontal space).  Useful in tabular environments.
\newcommand{\hangp}[1]{\makebox[0pt][r]{(}#1\makebox[0pt][l]{)}}

%%
% Prints an asterisk that takes up no horizontal space.
% Useful in tabular environments.
\newcommand{\hangstar}{\makebox[0pt][l]{*}}

%%
% Prints a trailing space in a smart way.
\usepackage{xspace}

%%
% Some shortcuts for Tufte's book titles.  The lowercase commands will
% produce the initials of the book title in italics.  The all-caps commands
% will print out the full title of the book in italics.
\newcommand{\vdqi}{\textit{VDQI}\xspace}
\newcommand{\ei}{\textit{EI}\xspace}
\newcommand{\ve}{\textit{VE}\xspace}
\newcommand{\be}{\textit{BE}\xspace}
\newcommand{\VDQI}{\textit{The Visual Display of Quantitative Information}\xspace}
\newcommand{\EI}{\textit{Envisioning Information}\xspace}
\newcommand{\VE}{\textit{Visual Explanations}\xspace}
\newcommand{\BE}{\textit{Beautiful Evidence}\xspace}

\newcommand{\TL}{Tufte-\LaTeX\xspace}

% Prints the month name (e.g., January) and the year (e.g., 2008)
\newcommand{\monthyear}{%
  \ifcase\month\or January\or February\or March\or April\or May\or June\or
  July\or August\or September\or October\or November\or
  December\fi\space\number\year
}


% Prints an epigraph and speaker in sans serif, all-caps type.
\newcommand{\openepigraph}[2]{%
  %\sffamily\fontsize{14}{16}\selectfont
  \begin{fullwidth}
  \sffamily\large
  \begin{doublespace}
  \noindent\allcaps{#1}\\% epigraph
  \noindent\allcaps{#2}% author
  \end{doublespace}
  \end{fullwidth}
}

% Inserts a blank page
\newcommand{\blankpage}{\newpage\hbox{}\thispagestyle{empty}\newpage}

\usepackage{units}

% Typesets the font size, leading, and measure in the form of 10/12x26 pc.
\newcommand{\measure}[3]{#1/#2$\times$\unit[#3]{pc}}

% Macros for typesetting the documentation
\newcommand{\hlred}[1]{\textcolor{Maroon}{#1}}% prints in red
\newcommand{\hangleft}[1]{\makebox[0pt][r]{#1}}
\newcommand{\hairsp}{\hspace{1pt}}% hair space
\newcommand{\hquad}{\hskip0.5em\relax}% half quad space
\newcommand{\TODO}{\textcolor{red}{\bf TODO!}\xspace}
\newcommand{\ie}{\textit{i.\hairsp{}e.}\xspace}
\newcommand{\eg}{\textit{e.\hairsp{}g.}\xspace}
\newcommand{\na}{\quad--}% used in tables for N/A cells
\providecommand{\XeLaTeX}{X\lower.5ex\hbox{\kern-0.15em\reflectbox{E}}\kern-0.1em\LaTeX}
\newcommand{\tXeLaTeX}{\XeLaTeX\index{XeLaTeX@\protect\XeLaTeX}}
% \index{\texttt{\textbackslash xyz}@\hangleft{\texttt{\textbackslash}}\texttt{xyz}}
\newcommand{\tuftebs}{\symbol{'134}}% a backslash in tt type in OT1/T1
\newcommand{\doccmdnoindex}[2][]{\texttt{\tuftebs#2}}% command name -- adds backslash automatically (and doesn't add cmd to the index)
\newcommand{\doccmddef}[2][]{%
  \hlred{\texttt{\tuftebs#2}}\label{cmd:#2}%
  \ifthenelse{\isempty{#1}}%
    {% add the command to the index
      \index{#2 command@\protect\hangleft{\texttt{\tuftebs}}\texttt{#2}}% command name
    }%
    {% add the command and package to the index
      \index{#2 command@\protect\hangleft{\texttt{\tuftebs}}\texttt{#2} (\texttt{#1} package)}% command name
      \index{#1 package@\texttt{#1} package}\index{packages!#1@\texttt{#1}}% package name
    }%
}% command name -- adds backslash automatically
\newcommand{\doccmd}[2][]{%
  \texttt{\tuftebs#2}%
  \ifthenelse{\isempty{#1}}%
    {% add the command to the index
      \index{#2 command@\protect\hangleft{\texttt{\tuftebs}}\texttt{#2}}% command name
    }%
    {% add the command and package to the index
      \index{#2 command@\protect\hangleft{\texttt{\tuftebs}}\texttt{#2} (\texttt{#1} package)}% command name
      \index{#1 package@\texttt{#1} package}\index{packages!#1@\texttt{#1}}% package name
    }%
}% command name -- adds backslash automatically
\newcommand{\docopt}[1]{\ensuremath{\langle}\textrm{\textit{#1}}\ensuremath{\rangle}}% optional command argument
\newcommand{\docarg}[1]{\textrm{\textit{#1}}}% (required) command argument
\newenvironment{docspec}{\begin{quotation}\ttfamily\parskip0pt\parindent0pt\ignorespaces}{\end{quotation}}% command specification environment
\newcommand{\docenv}[1]{\texttt{#1}\index{#1 environment@\texttt{#1} environment}\index{environments!#1@\texttt{#1}}}% environment name
\newcommand{\docenvdef}[1]{\hlred{\texttt{#1}}\label{env:#1}\index{#1 environment@\texttt{#1} environment}\index{environments!#1@\texttt{#1}}}% environment name
\newcommand{\docpkg}[1]{\texttt{#1}\index{#1 package@\texttt{#1} package}\index{packages!#1@\texttt{#1}}}% package name
\newcommand{\doccls}[1]{\texttt{#1}}% document class name
\newcommand{\docclsopt}[1]{\texttt{#1}\index{#1 class option@\texttt{#1} class option}\index{class options!#1@\texttt{#1}}}% document class option name
\newcommand{\docclsoptdef}[1]{\hlred{\texttt{#1}}\label{clsopt:#1}\index{#1 class option@\texttt{#1} class option}\index{class options!#1@\texttt{#1}}}% document class option name defined
\newcommand{\docmsg}[2]{\bigskip\begin{fullwidth}\noindent\ttfamily#1\end{fullwidth}\medskip\par\noindent#2}
\newcommand{\docfilehook}[2]{\texttt{#1}\index{file hooks!#2}\index{#1@\texttt{#1}}}
\newcommand{\doccounter}[1]{\texttt{#1}\index{#1 counter@\texttt{#1} counter}}


% Custom
\usepackage{listings}

\definecolor{mygreen}{rgb}{0,0.6,0}
\definecolor{mygray}{rgb}{0.5,0.5,0.5}
\definecolor{mymauve}{rgb}{0.58,0,0.82}
\definecolor{myblue}{rgb}{0.205,0.142,0.73}

\lstset{ %
  language=C++,
  basicstyle=\footnotesize,        % size of fonts used for the code
  upquote=true,
  %aboveskip={1.5\baselineskip},
  columns=fixed,
  showstringspaces=false,
  extendedchars=false,
  breaklines=true,                 % automatic line breaking only at whitespace
  prebreak = \raisebox{0ex}[0ex][0ex]{\ensuremath{\hookleftarrow}},
  frame=single,
  numbers=left,
  showtabs=false,
  showspaces=false,
  identifierstyle=\ttfamily,
  keywordstyle=\color{blue},       % keyword style
  commentstyle=\color{mygreen},    % comment style
  stringstyle=\color{mymauve},     % string literal style
  numberstyle=\color{myblue},
  backgroundcolor=\color{white},   % choose the background color
  escapeinside={\%*}{*)},          % if you want to add LaTeX within your code
  captionpos=b,                    % sets the caption-position to bottom
  tabsize=4,
  numberstyle=\tiny,
  numbersep=5pt,
  keywords=[2]{break,continue,return}, 
  keywordstyle={[2]\color{blue}},
  %stepnumber=5,
  %numberfirstline=false
}

\newcommand{\cpplistings}[2]{
  \renewcommand{\figurename}{Listing}
  %\afterpage{
  \begin{figure}[!p]
    \lstinputlisting[language=C++,caption={}]{evy/src/#1.cc}
    \caption{#2}
    \label{ch\thechapter/#1}
    \index{#2 in chapter \thechapter} % might use a separate index for that.
  \end{figure}
  %}
  \renewcommand{\figurename}{Figure}
}

\setcounter{totalnumber}{100}
\setcounter{bottomnumber}{100}
\setcounter{topnumber}{100}
\setcounter{dbltopnumber}{100}

% Generates the index
\usepackage{makeidx}
\makeindex

\begin{document}

% Front matter
\frontmatter

% r.1 blank page
\blankpage

% v.2 epigraphs
\newpage\thispagestyle{empty}
%\openepigraph{%
%The public is more familiar with bad design than good design.
%It is, in effect, conditioned to prefer bad design, 
%because that is what it lives with. 
%The new becomes threatening, the old reassuring.
%}{Paul Rand%, {\itshape Design, Form, and Chaos}
%}
%\vfill
%\openepigraph{%
%A designer knows that he has achieved perfection 
%not when there is nothing left to add, 
%but when there is nothing left to take away.
%}{Antoine de Saint-Exup\'{e}ry}
%\vfill
%\openepigraph{%
%\ldots the designer of a new system must not only be the implementor and the first 
%large-scale user; the designer should also write the first user manual\ldots 
%If I had not participated fully in all these activities, 
%literally hundreds of improvements would never have been made, 
%because I would never have thought of them or perceived 
%why they were important.
%}{Donald E. Knuth}


% r.3 full title page
\maketitle


% v.4 copyright page
\newpage
\begin{fullwidth}
~\vfill
\thispagestyle{empty}
\setlength{\parindent}{0pt}
\setlength{\parskip}{\baselineskip}
Copyright \copyright\ \the\year\ \thanklessauthor

%\par\smallcaps{Published by \thanklesspublisher}

%\par\smallcaps{tufte-latex.github.io/tufte-latex/}

%\par Licensed under the Apache License, Version 2.0 (the ``License''); you may not
%use this file except in compliance with the License. You may obtain a copy
%of the License at \url{http://www.apache.org/licenses/LICENSE-2.0}. Unless
%required by applicable law or agreed to in writing, software distributed
%under the License is distributed on an \smallcaps{``AS IS'' BASIS, WITHOUT
%WARRANTIES OR CONDITIONS OF ANY KIND}, either express or implied. See the
%License for the specific language governing permissions and limitations
%under the License.\index{license}

\par\textit{Version v0.3.0, \monthyear}
\end{fullwidth}

% r.5 contents
\tableofcontents

\listoffigures

\listoftables

% r.7 dedication
\cleardoublepage
~\vfill
\begin{doublespace}
\noindent\fontsize{18}{22}\selectfont\itshape
\nohyphenation
%Dedicated to those who appreciate \LaTeX{} 
%and the work of \mbox{Edward R.~Tufte} 
%and \mbox{Donald E.~Knuth}.
In memory of Jakub Jaro\v{s}.
\end{doublespace}
\vfill
\vfill


% r.9 introduction
\cleardoublepage
%\chapter*{Introduction}
 
%%
% Start the main matter (normal chapters)
\mainmatter

\newcommand\red{\color{red}}
\newcommand\blue{\color{blue}}
\newcommand\mygreen{\color{green}}

\chapter{Introduction}

\section{Difference coding}

\section{Run-length encoding}

\section{Front compression}


\chapter{Coding of integers}

Coding of integers is important prerequisite to data compression. Some of these codes are unable to directly encode $0$ and none of these codes are able to directly encode negative numbers. Generally, these codes can be classified as uniquely decodable and non-uniquely decodable.

\section{Unary codes}
Encoding\sidenote{$\alpha$-code is uniquely decodable (prefix) code with variable length. It does not have the ability to encode value $0$ or negative numbers directly, usually shift of values is introduced to handle such cases, for example encoding $x+1$ or $x+k, k\geq1$.}:

\subsection{$\alpha$-code}
$$\alpha(x)=0^{x-1}1$$
\subsection{$\alpha'$-code}
This variant has $0$ and $1$ swapped.\sidenote{This variant of $\alpha$-code is primarily used in Golomb Codes.}
$$\alpha'(x)=1^{x-1}0$$
\section{Binary codes}
Encoding\sidenote{$\beta$-codes are non-uniquely decodable codes with variable length. Fixed length binary codes are also widely used, they are uniquely decodable, but alsways encode to the same number of bits and they have a maximum value. $\beta'$-code has leading $1$ removed.}:
\subsection{$\beta$-code}
$$\beta(0)=0$$
$$\beta(1)=1$$
$$\beta(2i+j)=\beta(i)\beta(j)$$
\subsection{$\beta'$-code}
$$\beta'(x) = \beta(x)-10^{|\beta(x)|-1}$$

\section{Ternary codes}
These codes introduce a special symbol $\#$ for marking end of each keyword. \sidenote{$\tau$-codes are uniquely decodable (prefix) codes with variable length.}

\noindent
Encoding:

\subsection{$\tau$-code}
$$\tau(x) = \beta(x)\#$$
\subsection{$\tau'$-code}
$$\tau'(x) = \beta'(x)\#$$

\section{Elias codes}
Encoding\sidenote{$\gamma$-codes are uniquely decodable codes with variable length. Uniquely decodable $\alpha$-code is combined with non-uniquely decodable $\beta$-code. The difference between $\gamma'$-code and $\gamma$-code is that $\gamma$-code is interleaved. $\delta$-code is similar to $\gamma'$-code, but $\alpha$-code is replaced with $\gamma$-code.}:

\subsection{$\gamma'$-code}
$$\gamma'(x)=\alpha(|\beta(x)|)\beta'(x)$$

\subsection{$\gamma$-code}
$$y=\alpha(|\beta(x)|), z=\beta'(x)$$
$$\gamma(x)=y_1,z_1,y_2,z_2,\ldots,z_{|\beta(x)|-1},y_{|\beta(x)|}$$

\subsection{$\delta$-code}
$$\gamma(x)=\gamma(|\beta(x)|)\beta'(x)$$

\noindent
The $\omega$-codes have recursive prefix. The value is encoded by $\beta$-code followed with $0$. If the encoded $\beta$-code value is longer than $k, k \geq 2$ then a prefix is recursively prolonged by the prepending the value of $\beta$-code of the length of the previous part encoded by $\beta'$-code (length of $\beta$-code $-1$). If the length of $\beta$-code is shorted then $k$ then zerozes are prepended to make it of length $k$.

\noindent
Encoding\sidenote{$\omega$-codes are uniquely decodable codes with variable length. They do not have the ability to encode value $0$ or negative numbers directly}

\subsection{$\omega$-code ($k=2$)}
$$\omega(1) = 0^{k-1} = 0$$
$$\omega(x)= \text{<prefix>}\beta(x)0$$

\subsection{$\omega'$-code ($k=3$)}
$$\omega(1) = 0^{k-1} = 00$$
$$\omega(x)= \text{<prefix>}\beta(x)0$$

\section{Ternary Comma Code}
It is a simple code based on base 3 number representation and comma (c), which gives the following encoding:

\begin{table}
\begin{tabular}{|c|c|c|c|}\hline
    0 & 1 & 2 & c\\\hline
    00 & 01 & 10 & 11\\\hline
\end{tabular}
\caption{Ternary Comma Code}
\end{table}

\section{Fibbonacci Code}
Binary number is representation where each position corresponts to $2^i, i \in \mathbb{N}_0$. These powers of $2$ can be raplaced with numbers from Fibbonacci sequence, excluding the initial $1$:
$$1, 2, 3, 5, 8, 13, 21, 34\ldots$$.

This gives us multiple representations of each number as the two consecutive numbers can always be replaced with the next one, i.e. $11$ has the same meaning as $100$, for example: 
$$1*5+1*3=1*8+0*5+0*3.$$
Because of this, each number can be represented using this Fibbonacci representation as a sequeance of $0$ and $1$, where no two consecutive $1$ are present ($11$). When such number is reversed and additional $1$ is appended, it leads to a code, where each additional $1$ marks the end of the keyword.

\section{Golomb Code}
$GC(x,m)$ is Golomb Code of $x$ in modulo $m$.

\noindent
Quotient:
$$q=\floor{x/m}.$$
Reminder:
$$r=x-qm.$$
Coeficient:
$$c=\ceil{\log_2 m}.$$

\noindent
Reminder table $R$:\\
\begin{tabular}{r l l l l}
    $2^c-m$ & values of & $r$ & $\beta$-code encoded by $c-1$ & bits,\\
    $2m-2^c$ & values of & $r+2^c-m$ & $\beta$-code encoded by $c$ & bits.
\end{tabular}

\noindent
Encoding:
$$GC(x,m)=\alpha'(q+1)R(r).$$

\subsection{Examples of reminder tables}
$R$ for $m=3$:

\begin{table}
\begin{tabular}{|c||c|c|}\hline
    0 & 1 & 2\\\hline
    0 & 10 & 11\\\hline
\end{tabular}
\caption{Golomb Code $R$ for $m=3$}
\end{table}

\noindent
$R$ for $m=4$:

\begin{table}
\begin{tabular}{||c|c|c|c|}\hline
    0 & 1 & 2 & 3\\\hline
    00 & 01 & 10 & 11\\\hline
\end{tabular}
\caption{Golomb Code $R$ for $m=4$}
\end{table}

\noindent
$R$ for $m=5$:

\begin{table}
\begin{tabular}{|c|c|c||c|c|}\hline
    0 & 1 & 2 & 3 & 4\\\hline
    00 & 01 & 10 & 110 & 111\\\hline
\end{tabular}
\caption{Golomb Code $R$ for $m=5$}
\end{table}

\noindent
$R$ for $m=6$:

\begin{table}
\begin{tabular}{|c|c||c|c|c|c|}\hline
    0 & 1 & 2 & 3 & 4 & 5\\\hline
    00 & 01 & 100 & 101 & 110 & 111\\\hline
\end{tabular}
\caption{Golomb Code $R$ for $m=6$}
\end{table}

\noindent
$R$ for $m=7$:

\begin{table}
\begin{tabular}{|c||c|c|c|c|c|c|}\hline
    0 & 1 & 2 & 3 & 4 & 5 & 6\\\hline
    00 & 010 & 011 & 100 & 101 & 110 & 111\\\hline
\end{tabular}
\caption{Golomb Code $R$ for $m=7$}
\end{table}

\noindent
$R$ for $m=8$:

\begin{table}
\begin{tabular}{||c|c|c|c|c|c|c|c|}\hline
    0 & 1 & 2 & 3 & 4 & 5 & 6 & 7\\\hline
    000 & 001 & 010 & 011 & 100 & 101 & 110 & 111\\\hline
\end{tabular}
\caption{Golomb Code $R$ for $m=8$}
\end{table}

\noindent
$R$ for $m=14$:

\begin{table*}
\begin{tabular}{|c|c||c|c|c|c|c|c|c|c|c|c|c|c|}\hline
    0 & 1 & 2 & 3 & 4 & 5 & 6 & 7 & 8 & 9 & 10 & 11 & 12 & 13\\\hline
    000 & 001 & 0100 & 0101 & 0110 & 0111 & 1000 & 1001 & 1010 & 1011 & 1100 & 1101 & 1110 & 1111\\\hline
\end{tabular}
\\
\caption{Golomb Code $R$ for $m=14$}
\end{table*}

\section{Rice Code}
$RC(x,m) = GC(x,m)$ if $m = 2^k, k \in \mathbb{N}_0$. 

%\section{Properties of codes}

%Universal, Asymptotically optimal

%\section{Examples}

%\begin{table*}
%\begin{tabular}{|c|r|r|r|r|r|r|r|r|r|r|r|r|r|}\hline
%    x & $\alpha$-code & $\beta$-code & $\beta'$-code & $\gamma'$-code & $\gamma$-code & $\delta$-code & $\omega$-code & $\omega'$-code \\\hline
%    0  & -         & $0$      & -       & $\blue1$      & $\blue1$      & $\blue1$      & & \\
%    1  & $1$       & $1$      & $\varepsilon$ & & & & & \\
%    2  & $01$      & $10$     & $0$     & $\blue01\red0$     & $0$     & $0$     & & & \\
%    3  & $001$     & $11$     & $1$     & $\blue01\red1$     & $1$     & $1$     & & \\
%    4  & $0^31$    & $100$    & $00$    & $\blue001\red00$    & $00$    & $00$    & & \\
%    5  & $0^41$    & $101$    & $01$    & $\blue001\red01$    & $01$    & $01$    & & \\
%    6  & $0^51$    & $110$    & $10$    & $\blue001\red10$    & $10$    & $10$    & & \\
%    7  & $0^61$    & $110$    & $11$    & $\blue001\red10$    & $10$    & $10$    & & \\
%    8  & $0^71$    & $1000$   & $000$   & $\blue0001\red000$   & $000$   & $000$   & & \\
%    9  & $0^81$    & $1001$   & $001$   & $\blue0001\red001$   & $001$   & $001$   & & \\
%    10 & $0^91$    & $1010$   & $010$   & $\blue0001\red010$   & $010$   & $010$   & & \\
%    11 & $0^{10}1$ & $1011$   & $011$   & $\blue0001\red011$   & $011$   & $011$   & & \\
%    12 & $0^{11}1$ & $1100$   & $100$   & $\blue0001\red100$   & $100$   & $100$   & & \\
%    13 & $0^{12}1$ & $1101$   & $101$   & $\blue0001\red101$   & $101$   & $101$   & & \\
%    14 & $0^{13}1$ & $1110$   & $110$   & $\blue0001\red110$   & $110$   & $110$   & & \\
%    15 & $0^{14}1$ & $1111$   & $111$   & $\blue0001\red111$   & $111$   & $111$   & & \\
%    16 & $0^{15}1$ & $10000$  & $0000$  & $\blue00001\red0000$  & $0000$  & $0000$  & & \\
%    23 & $0^{22}1$ & $10111$  & $0111$  & $\blue00001\red0111$  & $0111$  & $0111$  & & \\
%    24 & $0^{23}1$ & $11000$  & $1000$  & $\blue00001\red1000$  & $1000$  & $1000$  & & \\
%    31 & $0^{30}1$ & $11111$  & $1111$  & $\blue00001\red1111$  & $1111$  & $1111$  & & \\
%    32 & $0^{31}1$ & $100000$ & $00000$ & $\blue000001\red00000$ & $00000$ & $00000$ & & \\\hline
%\end{tabular}
%\\
%\caption{Examples of coding of integers}
%\end{table*}

\chapter{Statistical methods I}

\section{Entrophy of source units}

Entrophy is the amount of information contained in a unit measurred in bits. The more unexpeted the unit is, the higher is the amount of information it carries.

There are several basic terms:

\noindent
Source units:
$$S = \{x_1, x_2, \ldots, x_n\}.$$
Source unit probabilities:
$$P = \{p_1, p_2, \ldots, p_n\}.$$
Source unit frequencies\sidenote{Frequencies are usually associated with actual messages and their analysis.}:
$$F = \{f_1, f_2, \ldots, f_n\}.$$
Entropy (information content) of unit $x_i$\sidenote{How much bits of actual information unit represents.}:
$$ H_i = - \log_2 p_i.$$
Average entrophy of a source unit from S\sidenote{How much bits of actual information represents a unit on average.}:
$$H_{avg}(S) = \sum_{i=1}^{n}{p_i H_i} = -\sum_{i=1}^{n}{p_i \log_2 p_i}.$$

%\section{Entrophy of a message}

%Message
%$$ X = x_1, x_2, \ldots, x_n,$$


%Entrophy of source message
%Entrophy of encdoded message
%Length of encoded message
%$$L(X) = $$

\section{Entrophy of a code}
To each source unit, a codeword can be assigned. 

\noindent
Codewords (code units):
$$C = \{c_1, c_2, \ldots, c_n\}.$$
Average length of a codeword\sidenote{How many actual bits are used for encoding a source unit into a codeword on average.}:
$$L_{avg}(C) = \sum_{i=1}^{n}{p_i |c_i|}.$$

\section{Propertis of codes}

Given the following source units, their probabilities and codewors, the average entrophy and the average length of the codeword can be calculated:

\noindent
Source units:
$$S = \{\texttt{a}, \texttt{b}, \texttt{c}, \texttt{d}, \texttt{e}\}.$$
Source unit probabilities:
$$P = \{0.1, 0.15, 0.3, 0.16, 0.29\}.$$
Codewords:
$$C = \{010, 011, 11, 00, 10\}.$$

\begin{table}
  \begin{center}
    \begin{tabular}{|r|cccccc|}
      \hline
      $x_i$ & $p_i$ & $c_i$ & $|c_i|$ & $H_i$ & $H_{avg}(S)$ & $L_{avg}(C)$      \\
          &        &        &          & $-\log_2 p_i$ & $p_i H_i$ & $p_i |c_i|$      \\
      %      & unit & probability & code & code length & unit entrophy & average entrophy & average code length      \\
      \hline
      a          & 0.10   & 010    & 3        & 3.32  & 0.332        & 0.30              \\  
      b          & 0.15   & 011    & 3        & 2.74  & 0.411        & 0.45              \\
      c          & 0.30   & 11     & 2        & 1.74  & 0.521        & 0.60              \\
      d          & 0.16   & 00     & 2        & 2.65  & 0.423        & 0.32              \\
      e          & 0.29   & 10     & 2        & 1.79  & 0.518        & 0.58              \\
      \hline
      $\sum$     & 1.00   & -      & -        & -     & 2.205        & 2.25              \\
      \hline
    \end{tabular}
  \end{center}
  \caption{Code $C$ for source units $S$ with probabilities $P$}
\end{table}

The closer is the average length of the codeword to the actual entrophy, the better is code performing.

%Prefix code - no code word is a prefix of the other code word.
%Uniqualy decodable (biunique) - no coded string has more than one decoding.
%Kraft's inequality - necessary and sufficient condition for the existence of a prefix code.


\begin{dt}{Prefix code.}
  A prefix code is a code where no codeword is a prefix of a different codeword.
\end{dt}

\begin{dt}{Biunique (uniquely decodable) code.}
  A biunique code is a code where no encoded message has more than one decoding.
\end{dt}

Code $C = \{010, 011, 11, 00, 10\}$ above is uniquely decodable prefix code as no codeword is a prefix of another codeword. Uniquely decodable code which is not a prefix code can be easily created by doing a reverse of each codeword of code $C$. $C^R = \{010, 110, 11, 00, 01\}$ is not a prefix code\sidenote{Code $C^R$ is actually a suffix code.} because $11$ is a prefix of $110$ and $01$ is a prefix of $010$.

Code $C_N = \{010, 011, 11, 00, 01\}$ is not uniquely decodable code as sequence $010011$ can be interpreted in several way as $010.011$ or $01.00.11$.

All prefix codes are uniquely decodable codes. Some uniquely decodable codes are prefix codes.\sidenote{In practice, mostly prefix codes are used.}

\section{Sample source units and their probabilities and frequencies}

These source units and their frequencies will be used in the following examples:

\noindent
Source units:
$$S_1 = \{\texttt{a}, \texttt{b}, \texttt{c}, \texttt{d}, \texttt{e}, \texttt{f}, \texttt{g}, \texttt{h}, \texttt{i}\}.$$
Source unit probabilities:
$$P_1 = \{1/35, 1/35, 2/35, 3/35, 3/35, 5/35, 5/35, 7/35, 8/35\}.$$
Source unit frequencies:
$$F_1 = \{1, 1, 2, 3, 3, 5, 5, 7, 8\}.$$

\forestset{angled/.style={content/.expanded={\noexpand\textless\forestov{content}\noexpand\textgreater}}}
\tikzset{el style/.style={midway, font=\scriptsize, inner sep=+1pt, auto=right}}
\tikzset{every node/.style={draw, circle}}
\tikzset{every edge/.style={align=center, base=top}}

\section{Shannon-Fano Coding}
Shannon-Fano Coding creates codewords by sorting units by their number of frequencies (probabilities). In each step the number of frequencies is split to two closest parts and new nodes are created. The nodes are created from the root which contains the sum of all frequencies.

\begin{figure}
\begin{forest}
  for tree={child anchor=north,inner sep=6pt},
  where n children={1}{tier=word}{},
  where n children={0}{rectangle,draw=none,minimum size=0.7cm}{
    if={n==1}{% n == 1 means first child
      edge label={node[el style, draw=none, swap, swap, near start]{0}}
    }{
      edge label={node[el style, swap, draw=none, near start]{1}}
    }
  }
%
[35 [ 15 [ 7 [ 4 [ 2 [ 1 [ a ] ]
                     [ 1 [ b ] ] ]
                 [ 2 [ c ] ] ]
             [ 3 [ d ] ] ]
         [ 8 [ 3 [ e ] ] 
             [ 5 [ f ] ] ] ]
    [ 20 [12 [ 5 [ g ] ]
             [ 7 [ h ] ] ]
          [8 [ i ] ] ] ]
\end{forest}
\caption{Shannon-Fano tree for code $C_1$ for source units $S$ with probabilities $P$}
\end{figure}

Edges to the left and to the right child nodes are labeled 0 and 1 respectively. The codewords can be red from root node to each leaf node.

From the tree above, the codewords are:
$$C_1 = \{\texttt{00000}, \texttt{00001}, \texttt{0001}, \texttt{001}, \texttt{010}, \texttt{011}, \texttt{100}, \texttt{101}, \texttt{11}\}.$$

\begin{table}
  \begin{center}
    \begin{tabular}{|r|cccccc|}
      \hline
      $x_i$ & $p_i$ & $c_i$ & $|c_i|$ & $H_i$ & $H_{avg}(S)$ & $L_{avg}(C)$      \\
          &        &        &          & $-\log_2 p_i$ & $p_i H_i$ & $p_i |c_i|$      \\
      \hline
      a          & 0.029   & 00000    & 5        & 5.13  & 0.147        & 0.143              \\  
      b          & 0.029   & 00001    & 5        & 5.13  & 0.147        & 0.143              \\
      c          & 0.057   & 0001     & 4        & 4.13  & 0.236        & 0.229              \\
      d          & 0.086   & 001      & 3        & 3.54  & 0.304        & 0.257              \\
      e          & 0.086   & 010      & 3        & 3.54  & 0.304        & 0.257              \\
      f          & 0.143   & 011      & 3        & 2.81  & 0.401        & 0.429              \\  
      g          & 0.143   & 100      & 3        & 2.81  & 0.401        & 0.429              \\
      h          & 0.200   & 101      & 3        & 2.32  & 0.464        & 0.600              \\
      i          & 0.229   & 11       & 2        & 2.19  & 0.487        & 0.457              \\
      \hline
      $\sum$     & 1.00    & -        & -        & -     & 2.891        & 2.944              \\
      \hline
    \end{tabular}
  \end{center}
  \caption{Code $C_1$ for source units $S_1$ with probabilities $P_1$}
\end{table}
  
Sometimes parts can be created differently, leading to different codewords. In our case, in the first step the number of frequencies can be split either to $1, 1, 2, 3, 3, 5$ and $5, 7, 8$ or  $1, 1, 2, 3, 3, 5, 5$ and $7, 8$ leading to $15 + 20$ or $20 + 15$. It has to be decided beforehand how to resolve this situation (either going with lower on left or lower on right).

\begin{figure}
\begin{forest}
  for tree={child anchor=north,inner sep=6pt},
  where n children={1}{tier=word}{},
  where n children={0}{rectangle,draw=none,minimum size=0.7cm}{
    if={n==1}{% n == 1 means first child
      edge label={node[el style, draw=none, swap, swap, near start]{0}}
    }{
      edge label={node[el style, swap, draw=none, near start]{1}}
    }
  }
%
[35 [ 20 [ 10 [ 4 [ 2 [ 1 [ a ] ]
                      [ 1 [ b ] ] ]
                  [ 2 [ c ] ] ]
              [ 6 [ 3 [ d ] ]
                  [ 3 [ e ] ] ] ]
         [ 10 [5 [ f ] ]
              [5 [ g ] ] ] ]
    [ 15 [ 7 [ h ] ]
         [ 8 [ i ] ] ] ]
\end{forest}
\caption{Huffman tree for code $C_2$ for source units $S$ with probabilities $P$}
\end{figure}

From the tree above, the codewords are:
$$C_2 = \{\texttt{00000}, \texttt{00001}, \texttt{0001}, \texttt{0010}, \texttt{0011}, \texttt{010}, \texttt{111}, \texttt{10}, \texttt{11}\}.$$

\begin{table}
  \begin{center}
    \begin{tabular}{|r|cccccc|}
      \hline
      $x_i$ & $p_i$ & $c_i$ & $|c_i|$ & $H_i$ & $H_{avg}(S)$ & $L_{avg}(C)$      \\
          &        &        &          & $-\log_2 p_i$ & $p_i H_i$ & $p_i |c_i|$      \\
      \hline
      a          & 0.029   & 00000    & 5        & 5.13  & 0.147        & 0.143              \\  
      b          & 0.029   & 00001    & 5        & 5.13  & 0.147        & 0.143              \\
      c          & 0.057   & 0001     & 4        & 4.13  & 0.236        & 0.229              \\
      d          & 0.086   & 0010     & 4        & 3.54  & 0.304        & 0.343              \\
      e          & 0.086   & 0011     & 4        & 3.54  & 0.304        & 0.343              \\
      f          & 0.143   & 010      & 3        & 2.81  & 0.401        & 0.429              \\  
      g          & 0.143   & 111      & 3        & 2.81  & 0.401        & 0.429              \\
      h          & 0.200   & 10       & 2        & 2.32  & 0.464        & 0.400              \\
      i          & 0.229   & 11       & 2        & 2.19  & 0.487        & 0.457              \\
      \hline
      $\sum$     & 1.00    & -        & -        & -     & 2.891        & 2.916              \\
      \hline
    \end{tabular}
  \end{center}
  \caption{Code $C_2$ for source units $S_1$ with probabilities $P_1$}
\end{table}

\section{Huffman Coding}
Huffman Coding creates codewords by creating new node from the two smallest nodes (sorting is usually involved). In each step the two lowest numbers of frequencies are summed together and a new node is created containing this sum as a parent of these two nodes. The nodes are created from the leaves which are created from the frequencies of the source units.

\begin{figure}
\begin{forest}
  for tree={child anchor=north,inner sep=6pt},
  where n children={1}{tier=terminus}{},
  where n children={0}{rectangle,draw=none,minimum size=0.7cm}{
  }
%i
[35 [ 20,edge label={node[el style, draw=none]{1}},name=N16 [9,edge label={node[el style, draw=none]{0}},name=N9 [ 4,edge label={node[el style, draw=none]{0}} [ 2,edge label={node[el style, draw=none]{0}} [ 1,edge label={node[el style, draw=none]{0}} [ a ] ]
                   [ 1,edge label={node[el style, draw=none, swap]{1}} [ b ] ] ]
                 [ 2,edge label={node[el style, draw=none, swap]{1}} [ c ] ] ]
                 [\phantom{0},draw=none,no edge] ] 
    [11,edge label={node[el style, draw=none, swap]{1}} [ 6,edge label={node[el style, draw=none, near start]{1}} [ 3,edge label={node[el style, draw=none]{0}} [ d ] ]
                 [ 3,edge label={node[el style, draw=none, swap, near end]{1}} [ e ] ] ]
             [ 5,name=N5,no edge [ f ] ]
             [ 5,edge label={node[el style, draw=none, swap, near start]{0}} [ g ] ] ] ]
    [ 15,edge label={node[el style, draw=none, swap]{0}} 
         [ 7,edge label={node[el style, draw=none]{0}} [ h ] ]
         [ 8,edge label={node[el style, draw=none, swap]{1}} [ i ] ] ] ]
  \draw (N9.south east) -- (N5.north) node[el style, swap, draw=none, near start]{1}; % or use (NP.north) 
\end{forest}
\caption{Huffman tree for code $C_3$ for source units $S$ with probabilities $P$}
\end{figure}

Edges to the lower and higher value child nodes are labeled 0 and 1 respectively. If a tie occures left and right child nodes are labeled 0 and 1 respectively.

From the tree above, the codewords are:
$$C_3 = \{\texttt{10000}, \texttt{10001}, \texttt{1001}, \texttt{1110}, \texttt{1111}, \texttt{101}, \texttt{110}, \texttt{00}, \texttt{01}\}.$$

\begin{table}
  \begin{center}
    \begin{tabular}{|r|cccccc|}
      \hline
      $x_i$ & $p_i$ & $c_i$ & $|c_i|$ & $H_i$ & $H_{avg}(S)$ & $L_{avg}(C)$      \\
          &        &        &          & $-\log_2 p_i$ & $p_i H_i$ & $p_i |c_i|$      \\
      \hline
      a          & 0.029   & 10000    & 5        & 5.13  & 0.147        & 0.143              \\  
      b          & 0.029   & 10001    & 5        & 5.13  & 0.147        & 0.143              \\
      c          & 0.057   & 1001     & 4        & 4.13  & 0.236        & 0.229              \\
      d          & 0.086   & 1110     & 4        & 3.54  & 0.304        & 0.343              \\
      e          & 0.086   & 1111     & 4        & 3.54  & 0.304        & 0.343              \\
      f          & 0.143   & 101      & 3        & 2.81  & 0.401        & 0.429              \\  
      g          & 0.143   & 110      & 3        & 2.81  & 0.401        & 0.429              \\
      h          & 0.200   & 00       & 2        & 2.32  & 0.464        & 0.400              \\
      i          & 0.229   & 01       & 2        & 2.19  & 0.487        & 0.457              \\
      \hline
      $\sum$     & 1.00    & -        & -        & -     & 2.891        & 2.916              \\
      \hline
    \end{tabular}
  \end{center}
  \caption{Code $C_3$ for source units $S_1$ with probabilities $P_1$}
\end{table}

\begin{dt}{Kraft-McMillan's Inequality.}
    The Kraft-McMillan's Inequality defined as\sidenote{The Kraft-McMillan's Inequality is derived from the Huffman tree, each leaf represents a codeword and the depth of the leaf correspond to the length of that codeword.}: $$\sum_{\ell \in \mathrm{leaves}} 2^{-\mathrm{depth}(\ell)} \leq 1.$$
    It is necessary and sufficient condition for the existence of a prefix code with given properties. If properties of the code given follows Kraft-McMillan's Inequality, it can be uniquely decodable code. If properties of the code given does not follow Kraft-McMillan's Inequality, it cannot be uniquely decodable code.
\end{dt}

Code $C_N = \{010, 011, 11, 00, 01\}$ above gives you $1/8 + 1/8 + 1/4 + 1/4 + 1/4 = 1$ which means that code with such length of codeword could be uniquely decodable code, but it is not.
Code $C_{N2} = \{00, 01, 10, 11, 000\}$ gives you $1/4 + 1/4 + 1/4 + 1/4 = 1.125$ hence it cannot be uniquely decodable code.

\begin{dt}{Redundant code.}
  A prefix code for which holds $$\sum_{\ell \in \mathrm{leaves}} 2^{-\mathrm{depth}(\ell)} < 1.$$
\end{dt}

Code $C_{R} = \{00, 01, 10, 110\}$ gives you $1/4 + 1/4 + 1/4 + 1/8 = 9.875$ hence the code is redundant (as it is also uniquely decodable).

\begin{dt}{Complete code.}
  A prefix code for which holds $$\sum_{\ell \in \mathrm{leaves}} 2^{-\mathrm{depth}(\ell)} = 1.$$
\end{dt}

Code $C = \{010, 011, 11, 00, 10\}$ above gives you $1/8 + 1/8 + 1/4 + 1/4 + 1/4 = 1$ hence the code is complete (as it is also uniquely decodable).

\begin{dt}{Optimal code.}
    A code is optimal (for a given probability distribution) if no other code with a lower average length of a codeword $L_{avg}$ exists.
\end{dt}

You can compare average length of codewords in examples above for Shannon-Fano Codes $C_1$ and $C_2$ and Huffman Code $C_3$. Huffman Coding always produces optimal code ($C_3$). Shanno-Fano Coding is not guaranteed to produce optimal code ($C_1$), but can produce optimal code ($C_2$). 

Optimal code is always complete code.

\section{Homework}

\begin{itemize}
  \item Try to create Shannon-Fano Coding and Huffman Coding for different sets of frequencies.
  \item Try to calculate Kraft-McMillan's Inequality for the created sets of codewords (included those presented here).
  \item Compare average codeword length of the created sets of coders and see that Shannon-Fano does not always produce an optimal code.
\end{itemize}

\chapter{Statistical methods II}
\section{Aritmetic coding}

\chapter{Dictionary methods}

\section{LZ77}

The LZ77 is based on the idea of sliding window divided to two parts, search buffer (SB) and lookahead buffer (LB).
Current position is always between SB and LB and is initialized on the first character of the string. In each iteration LZ77 produces a triplet $(\texttt{i}, \texttt{j}, \texttt{a})$ by searching longest possible prefix of LB within SB, where:
\begin{itemize}
  \item[\texttt{i}] is an index (counted from current position to the left),
  \item[\texttt{j}] is a count (number of matching symbols),
  \item[\texttt{a}] is a following symbol (first different symbol afther the match).
\end{itemize}
The dot will be used to mark current position and we consider $|SB|=6$ and $|LB|=4$.

\subsection{Examples}

Given
$$T = \texttt{aabaabaaabaaabaa}$$,
the algorithm proceeds as follows.
$$T = \texttt{.aabaabaaabaaabaa}$$
$$ \texttt{(0, 0, a)} $$
$$T = \texttt{a.abaabaaabaaabaa}$$
$$ \texttt{(0, 1, b)} $$
$$T = \texttt{aab.aabaaabaaabaa}$$
$$ \texttt{(2, 3, a)} $$
$$T = \texttt{aabaaba.aabaaabaa}$$
$$ \texttt{(3, 3, a)} $$
$$T = \texttt{aabaabaaaba.aabaa}$$
$$ \texttt{(3, 3, a)} $$
$$T = \texttt{aabaabaaabaaaba.a}$$
$$ \texttt{(0, 0, a)} $$
$$T = \texttt{aabaabaaabaaabaa.}$$
The following symbol can never be empty, so even if there would a match for it in the SB, is is not going to be used.
Result: $$ \texttt{(0, 0, a), (0, 1, b), (2, 3, a), (3, 3, a), (3, 3, a), (0, 0, a)} $$

Given
$$T = \texttt{abababcacacababa}$$,
the algorithm proceeds as follows.
$$T = \texttt{.abababcacacababa}$$
$$ \texttt{(0, 0, a)} $$
$$T = \texttt{a.bababcacacababa}$$
$$ \texttt{(0, 0, b)} $$
$$T = \texttt{ab.ababcacacababa}$$
$$ \texttt{(1, 3, b)} $$
Notice this special case, where we actually cross the boundary between SB and LB. It can be done because each additional symbol will be known during the decompression.
$$T = \texttt{ababab.cacacababa}$$
$$ \texttt{(0, 0, c)} $$
$$T = \texttt{abababc.acacababa}$$
$$ \texttt{(2, 1, c)} $$
$$T = \texttt{abababcac.acababa}$$
$$ \texttt{(1, 3, b)} $$
Again similar case as before.
$$T = \texttt{abababcacacab.aba}$$
$$ \texttt{(1, 2, a)} $$
$$T = \texttt{abababcacacababa.}$$
Result: $$ \texttt{(0, 0, a), (0, 0, b), (1, 3, b), (0, 0, c), (2, 1, c), (1, 3, b), (1, 2, a)} $$

Decompression is follows the same pattern and it is actually simpler, because there is no need to search for the best match. The only think done in each step is the copying symbols from SB (and possibly LB) into the LB.

You can examine the special case above during decompression, consider this state:
$$T = \texttt{abababcac.}$$,
where you want to add the following triplet:
$$ \texttt{(1, 3, b)} $$
You need to copy $3$ symbols from possition $1$ and the append \texttt{b} resulting in this:
$$T = \texttt{abababcac.acab}$$.
As you can see, the \texttt{ac} was copied from SB and then \texttt{a} from LB and then it was followed with \texttt{a} as the following symbol.

\section{LZ78}

In LZ78 the sliding window is replaced with standard trie like dictionary where nodes are numbered from $0$ incrementally and their number is used as reference. In each iteration algorithm produces a pair $(\texttt{i}, \texttt{a})$ by searching longest possible prefix from current position in the dictionary, where:
\begin{itemize}
  \item[\texttt{i}] is an index into the dictionary,
  \item[\texttt{a}] is a following symbol.
\end{itemize}
The dictionary is extended after each iteration by the processed string.

\clearpage
\subsection{Examples}

Given
$$T = \texttt{aabaabaaabaaabaa}$$, 
the algorithm stars with empty tree containing only root node which represents empty string $\varepsilon$ with index $0$.

\begin{marginfigure}
\begin{forest}
  for tree={child anchor=north,inner sep=5pt},
%
[0]
\end{forest}
\end{marginfigure}

$$ \texttt{(0, a)} $$

Node 1 for \texttt{a} is added.

\begin{marginfigure}
\begin{forest}
  for tree={child anchor=north,inner sep=5pt},
%
[0 [1,edge label={node[el style, draw=none]{a}}]]
\end{forest}
\end{marginfigure}

$$ \texttt{(1, b)}$$

Node 2 for \texttt{ab} is added.

\begin{marginfigure}[-1.5cm]
\hspace{1cm}
\begin{forest}
  for tree={child anchor=north,inner sep=5pt},
%
[0 [1,edge label={node[el style, draw=none]{a}} [2,edge label={node[el style, draw=none]{b}}]]]
\end{forest}
\end{marginfigure}

$$ \texttt{(1, a)}$$

Node 3 for \texttt{aa} is added.

\begin{marginfigure}[-1.5cm]
\hspace{2cm}
\begin{forest}
  for tree={child anchor=north,inner sep=5pt},
%
[0 [1,edge label={node[el style, draw=none]{a}} [2,edge label={node[el style, draw=none]{b}}]
                                                [3,edge label={node[el style, draw=none]{a}}]]]
\end{forest}
\end{marginfigure}

$$ \texttt{(0, b)}$$

Node 4 for \texttt{b} is added.

\begin{marginfigure}[-1.3cm]
\hspace{3.25cm}
\begin{forest}
  for tree={child anchor=north,inner sep=5pt},
%
[0 [1,edge label={node[el style, draw=none]{a}} [2,edge label={node[el style, draw=none]{b}}]
                                                [3,edge label={node[el style, draw=none]{a}}]]
   [4,edge label={node[el style, draw=none]{b}}]]
\end{forest}
\end{marginfigure}

$$ \texttt{(3, a)}$$

Node 5 for \texttt{aaa} is added.

\begin{marginfigure}[-1.3cm]
\hspace{4.4cm}
\begin{forest}
  for tree={child anchor=north,inner sep=5pt},
%
[0 [1,edge label={node[el style, draw=none]{a}} [2,edge label={node[el style, draw=none]{b}}]
                                                [3,edge label={node[el style, draw=none]{a}} [5,edge label={node[el style, draw=none]{a}}]]]
   [4,edge label={node[el style, draw=none]{b}}]]
\end{forest}
\end{marginfigure}

$$ \texttt{(4, a)}$$

Node 6 for \texttt{ba} is added.

\begin{marginfigure}[-2.4cm]
\hspace{2.25cm}
\begin{forest}
  for tree={child anchor=north,inner sep=5pt},
%
[0 [1,edge label={node[el style, draw=none]{a}} [2,edge label={node[el style, draw=none]{b}}]
                                                [3,edge label={node[el style, draw=none]{a}} [5,edge label={node[el style, draw=none]{a}}]]]
   [4,edge label={node[el style, draw=none]{b}} [6,edge label={node[el style, draw=none]{a}}]]]
\end{forest}
\end{marginfigure}

$$ \texttt{(3, b)}$$

Node 7 for \texttt{aab} is added.

\begin{marginfigure}[-2.6cm]
\begin{forest}
  for tree={child anchor=north,inner sep=5pt},
%
[0 [1,edge label={node[el style, draw=none]{a}} [2,edge label={node[el style, draw=none]{b}}]
                                                [3,edge label={node[el style, draw=none]{a}} [5,edge label={node[el style, draw=none]{a}}] 
                                                                                             [7,edge label={node[el style, draw=none]{b}}]]]
   [4,edge label={node[el style, draw=none]{b}} [6,edge label={node[el style, draw=none]{a}}]]]
\end{forest}
  
  \caption{$LZ78(\text{aabaabaaabaaabaa})$}
\end{marginfigure}

$$ \texttt{(1, a)}$$

Result: $$ \texttt{(0, a), (1, b), (1, a), (0, b), (3, a), (4, a), (3, b), (1, a)}$$

Given
$$T = \texttt{abababcacacababa}$$,
the algorithm proceeds similarly as with the previous example.

\begin{marginfigure}
\begin{forest}
  for tree={child anchor=north,inner sep=5pt},
%
  [0 [1,edge label={node[el style, draw=none]{a}} [3,edge label={node[el style, draw=none]{b}} [4,edge label={node[el style, draw=none]{c}}]]
                                                  [5,edge label={node[el style, draw=none]{c}} [5,edge label={node[el style, draw=none]{a}}]]]
     [2,edge label={node[el style, draw=none]{b}} [7,edge label={node[el style, draw=none]{a}}]]]
\end{forest}

  \caption{$LZ78(\text{abababcacacababa})$}

\end{marginfigure}

Result: $$ \texttt{(0, a), (0, b), (1, b), (3, c), (1, c), (5, a), (2, a), (2, a)}$$

The decompression proceeds simiarly by building an index using trie or table.

Given: $$ \texttt{(0, a), (1, b), (1, a), (0, b), (3, a), (4, a), (3, b), (1, a)}$$
the algorithm stars with empty table containing only empty string $\varepsilon$ with index $0$.

In each step you process a pair from the compressed input and add new index into dictionary. The encoded pair is decoded based on the previously added values.

\begin{figure}
  \begin{center}
  \begin{tabular}{c|l|l}
    index & enc. phrase & dec. phrase \\
    \hline
    0 & $\varepsilon$ & $\varepsilon$\\
    1 & 0a & \texttt{a}\\
    2 & 1b & \texttt{ab}\\
    3 & 1a & \texttt{aa}\\
    4 & 0b & \texttt{b}\\
    5 & 3a & \texttt{aaa}\\
    6 & 4a & \texttt{ba}\\
    7 & 3b & \texttt{aab}\\
    8 & 1a & \texttt{aa}\\
  \end{tabular}
  \end{center}
  \caption{$LZ78^R(LZ78(\text{aabaabaaabaaabaa}))$}
\end{figure}

Given: $$ \texttt{(0, a), (0, b), (1, b), (3, c), (1, c), (5, a), (2, a), (2, a)}$$
the algorithm stars with empty table containing only empty string $\varepsilon$ with index $0$.

In each step you process a pair from the compressed input and add new index into dictionary. The encoded pair is decoded based on the previously added values.

\begin{figure}
  \begin{center}
  \begin{tabular}{c|l|l}
    index & enc. phrase & dec. phrase \\
    \hline
    0 & $\varepsilon$ & $\varepsilon$\\
    1 & 0a & \texttt{a}\\
    2 & 0b & \texttt{b}\\
    3 & 1b & \texttt{ab}\\
    4 & 3c & \texttt{abc}\\
    5 & 1c & \texttt{ac}\\
    6 & 5a & \texttt{aca}\\
    7 & 2a & \texttt{ba}\\
    8 & 2a & \texttt{ba}\\
  \end{tabular}
  \end{center}
  \caption{$LZ78^R(LZ78(\text{abababcacacababa}))$}
\end{figure}

This method has no special cases.

\section{LZW}

LZW behaves similarly as LZ78, but the following symbol is dropped and in each iteration algorithm produces just the indes into the dictionary \texttt{i}. The dictionary is not initialized empty, but contains all characters of the alphabet instead.
The dictionary is extended after each iteration by the processed string and a following symbol. You can think about it that the following character is encoded this way. That trere is a overlap between string added into the dictionary then next processed string.
We consider alphabet $\Sigma = \{\texttt{a}, \texttt{b}, \texttt{c}\}$.

\begin{figure}
\begin{center}
\begin{forest}
  for tree={child anchor=north,inner sep=5pt},
%
  [0 [1,edge label={node[el style, draw=none]{a}}]                                         
     [2,edge label={node[el style, draw=none]{b}}]
     [3,edge label={node[el style, draw=none]{c}}]]
\end{forest}
\end{center}
  \caption{LZW - newly initialized dictionary}
\end{figure}

Given
$$T = \texttt{aabaabaaabaaabaa}$$,
the algorithm proceeds as follows.

\begin{marginfigure}
\begin{forest}
  for tree={child anchor=north,inner sep=5pt},
%
  [0 [1,edge label={node[el style, draw=none]{a}} [4,edge label={node[el style, draw=none]{a}} [7,edge label={node[el style, draw=none]{b}} [9,edge label={node[el style, draw=none]{a}}]] [10,edge label={node[el style, draw=none]{a}}]]
                                                  [5,edge label={node[el style, draw=none]{b}} [11,edge label={node[el style, draw=none]{a}}]]]                                         
     [2,edge label={node[el style, draw=none]{b}} [6,edge label={node[el style, draw=none]{a}} [8,edge label={node[el style, draw=none]{a}}]]]
     [3,edge label={node[el style, draw=none]{c}}]]
\end{forest}

  \caption{$LZW(\text{aabaabaaabaaabaa})$}
\end{marginfigure}

Result: $$ \texttt{1, 1, 2, 4, 6, 7, 4, 5, 4}$$

Given
$$T = \texttt{abababcacacababa}$$,
the algorithm proceeds as follows.

\begin{marginfigure}
\begin{forest}
  for tree={child anchor=north,inner sep=5pt},
%
  [0 [1,edge label={node[el style, draw=none]{a}} [4,edge label={node[el style, draw=none]{b}} [6,edge label={node[el style, draw=none]{a}}] [7,edge label={node[el style, draw=none]{c}}]]
                                                  [9,edge label={node[el style, draw=none]{c}}]]                                         
     [2,edge label={node[el style, draw=none]{b}} [5,edge label={node[el style, draw=none]{a}} [12,edge label={node[el style, draw=none]{b}}]]]
     [3,edge label={node[el style, draw=none]{c}} [8,edge label={node[el style, draw=none]{a}} [10,edge label={node[el style, draw=none]{c}}] [11,edge label={node[el style, draw=none]{b}}]]]]
\end{forest}

  \caption{$LZW(\text{aabaabaaabaaabaa})$}
\end{marginfigure}

Result: $$ \texttt{1, 2, 4, 4, 3, 1, 8, 8, 5, 5}$$

The decompression proceeds simiarly by building an index using trie or table. After second number is decoded adding into dictionary starts. The phrases added is alway the previous number plus the first carracter or decoded current number.

Given: $$ \texttt{1, 1, 2, 4, 6, 7, 4, 5, 4}$$
the algorithm stars with table containing empty string $\varepsilon$ with index $0$ and all symbols of the alphabet with increasing indexes.

In each step you process a pair from the compressed input and add new index into dictionary. The encoded pair is decoded based on the previously added values.

\begin{figure*}
  \begin{center}
  \begin{tabular}{c|l|l}
    index & enc. phrase & dec. phrase \\
    \hline
    0 & $\varepsilon$ & $\varepsilon$\\
    1 & a & \texttt{a}\\
    2 & b & \texttt{b}\\
    3 & c & \texttt{c}\\
    4 & 1a & \texttt{aa}\\
    5 & 1b & \texttt{ab}\\
    6 & 2a & \texttt{ba}\\
    7 & 4b & \texttt{aab}\\
    8 & 6a & \texttt{baa}\\
    9 & 7a & \texttt{aaba}\\
    10 & 4a & \texttt{aaa}\\
    11 & 5a & \texttt{aba}\\
  \end{tabular}
  \begin{tabular}{|c|c|c|c|c|c|c|c|c|}
    \hline
    1 & 1 & 2 & 4 & 6 & 7 & 4 & 5 & 4\\
    \hline
    a & a & b & aa & ba & aab & aa & ab & aa\\
    \hline
  \end{tabular}
  \end{center}
  \caption{$LZWW^R(LWZ(\text{aabaabaaabaaabaa}))$}
\end{figure*}

Given: $$ \texttt{1, 2, 4, 4, 3, 1, 8, 8, 5, 5}$$
the algorithm stars with table containing empty string $\varepsilon$ with index $0$ and all symbols of the alphabet with increasing indexes.

In each step you process a pair from the compressed input and add new index into dictionary. The encoded pair is decoded based on the previously added values.

\begin{figure*}
  \begin{center}
  \begin{tabular}{c|l|l}
    index & enc. phrase & dec. phrase \\
    \hline
    0 & $\varepsilon$ & $\varepsilon$\\
    1 & a & \texttt{a}\\
    2 & b & \texttt{b}\\
    3 & c & \texttt{c}\\
    4 & 1b & \texttt{ab}\\
    5 & 2a & \texttt{ba}\\
    6 & 4a & \texttt{aba}\\
    7 & 4c & \texttt{abc}\\
    8 & 3a & \texttt{ca}\\
    9 & 1c & \texttt{ac}\\
    10 & 8c & \texttt{cac}\\
    11 & 8b & \texttt{cab}\\
    12 & 5b & \texttt{bab}\\
  \end{tabular}
  \begin{tabular}{|c|c|c|c|c|c|c|c|c|c|}
    \hline
    1 & 2 & 4 & 4 & 3 & 1 & 8 & 8 & 5 & 5\\
    \hline
    a & b & ab & ab & c & a & ca & ca & ba & ba\\
    \hline
  \end{tabular}
  \end{center}
  \caption{$LZW^R(LZW(\text{abababcacacababa}))$}
\end{figure*}

The algorithm has one special case - when the currently encoded string is a prefix minus $1$ character from the next string. If this case does not concerd one of the initial nodes, it is represented by a node which parent has index minus $1$.

Given
$$T = \texttt{abbbbabc}$$,
the algorithm proceeds as follows..

\begin{marginfigure}
\begin{forest}
  for tree={child anchor=north,inner sep=5pt},
%
  [0 [1,edge label={node[el style, draw=none]{a}} [4,edge label={node[el style, draw=none]{b}}]]                                         
     [2,edge label={node[el style, draw=none]{b}} [5,edge label={node[el style, draw=none]{b}} [7,edge label={node[el style, draw=none]{b}}]] [6,edge label={node[el style, draw=none]{a}}]]
     [3,edge label={node[el style, draw=none]{c}}]]
\end{forest}

  \caption{$LZW(\text{abbbbabc})$}
\end{marginfigure}

Result: $$ \texttt{1, 2, 5, 2, 4, 3}$$

Given: $$ \texttt{1, 2, 5, 2, 4, 3}$$
the algorithm stars with table containing empty string $\varepsilon$ with index $0$ and all symbols of the alphabet with increasing indexes.

In each step you process a pair from the compressed input and add new index into dictionary. The encoded pair is decoded based on the previously added values.

\begin{figure}
  \begin{center}
  \begin{tabular}{c|l|l}
    index & enc. phrase & dec. phrase \\
    \hline
    0 & $\varepsilon$ & $\varepsilon$\\
    1 & a & \texttt{a}\\
    2 & b & \texttt{b}\\
    3 & c & \texttt{c}\\
    4 & 1b & \texttt{ab}\\
    5 & 5b & \texttt{bb}\\
    6 & 5b & \texttt{bbb}\\
    7 & 2a & \texttt{ba}\\
  \end{tabular}
  \begin{tabular}{|c|c|c|c|c|c|}
    \hline
    1 & 2 & 5 & 2 & 4 & 3\\
    \hline
    a & b & bb & b & ab & c\\
    \hline
  \end{tabular}
  \end{center}
  \caption{$LZW^R(LZW(\text{abbbbabc}))$}
\end{figure}

\section{Homework}

\begin{itemize}
  \item Try to figure out efficient way how to encode outputs from LZ77, LZ78 and LZW methods. 
  \item Try to figure out how many bits you will need for the encoding.
  \item Try to calculate compression rations. 
\end{itemize}

\chapter{Context methods I}

Context methods are a group of methods of compression which exploits the different probablities of symbols based on the preceeding characters - the context. Each of these methods usually works in a completely different manner, but the basis idea about context is common to all of them.

\section{Entrophy of higher order}

Entrophy of higher order is closely related to context methods. Order is length of the context preceeding the character. When is order $0$ it means that the context is empty and it is just and ordinary entrophy.

\begin{dt}{Entrophy of $0$ order, $T = \Sigma^+$.}
  The entrophy of $0$ order is defined $$H_{0}(T) = - \sum_{a \in \Sigma} \frac{|T_a|}{n} \log_2 \frac{|T_a|}{n}.$$
\end{dt}

\begin{dt}{Entrophy of $k$ order, $k > 0$.}
  The entrophy of $k$ order is defined $$H_{k}(T) = \frac{1}{n} \sum_{w \in \Sigma^k} |w_T| H_0(w_T).$$
\end{dt}

\begin{figure}
    $$H_{0}(\text{aabaabaaabaaabaa}) = - (\frac{12}{16} \log_2 \frac{12}{16} + \frac{4}{16} \log_2 \frac{4}{16})$$
  \caption{$H_{0}(\text{aabaabaaabaaabaa})$}
\end{figure}

\begin{figure}
  \begin{center}
  \begin{tabular}{c|c|c}
    $w$ & $w_T$ & $|w_T|$ \\
    \hline
    a & ababaabaaba & 11 \\
    b & aaaa & 4 \\
  \end{tabular}
  \end{center}
    $$H_{1}(\text{aabaabaaabaaabaa}) = \frac{1}{16} (11 H_{0}(\text{ababaabaaba}) + 4 H_{0}(\text{aaaa}))$$
  \caption{$H_{1}(\text{aabaabaaabaaabaa})$}
\end{figure}

\begin{figure}
  \begin{center}
  \begin{tabular}{c|c|c}
    $w$ & $w_T$ & $|w_T|$ \\
    \hline
    aa & bbabab & 5 \\
    ab & aaaa & 4 \\
    ba & aaaa & 4 \\
    bb & $\varepsilon$ & 0 \\
  \end{tabular}
  \end{center}
    $$H_{2}(\text{aabaabaaabaaabaa}) = \frac{1}{16} (5 H_{0}(\text{bbabab}) + 4 H_{0}(\text{aaaa}) + 4 H_{0}(\text{aaaa}))$$
  \caption{$H_{2}(\text{aabaabaaabaaabaa})$}
\end{figure}

\begin{figure}
  \begin{center}
  \begin{tabular}{c|c|c}
    $w$ & $w_T$ & $|w_T|$ \\
    \hline
    aaa & bb & 2 \\
    aab & aaaa & 4 \\
    aba & aaaa & 4 \\
    abb & $\varepsilon$ & 0 \\
    baa & baa & 3 \\
    bab & $\varepsilon$ & 0 \\
    bba & $\varepsilon$ & 0 \\
    bbb & $\varepsilon$ & 0 \\
  \end{tabular}
  \end{center}
    $$H_{3}(\text{aabaabaaabaaabaa}) = \frac{1}{16} (2 H_{0}(\text{bb}) + 4 H_{0}(\text{aaaa}) + 4 H_{0}(\text{aaaa}) + 3 H_{0}(\text{baa}))$$
  \caption{$H_{3}(\text{aabaabaaabaaabaa})$}
\end{figure}

\begin{figure}
  \begin{center}
  \begin{tabular}{c|c|c}
    $w$ & $w_T$ & $|w_T|$ \\
    \hline
    aaaa & $\varepsilon$ & 0 \\
    aaab & aa & 2 \\
    aaba & aaaa & 4 \\
    aabb & $\varepsilon$ & 0 \\
    abaa & baa & 3 \\
    abab & $\varepsilon$ & 0 \\
    abba & $\varepsilon$ & 0 \\
    abbb & $\varepsilon$ & 0 \\
    baaa & bb & 2 \\
    baab & a & 1 \\
    baba & $\varepsilon$ & 0 \\
    babb & $\varepsilon$ & 0 \\
    bbaa & $\varepsilon$ & 0 \\
    bbab & $\varepsilon$ & 0 \\
    bbba & $\varepsilon$ & 0 \\
    bbbb & $\varepsilon$ & 0 \\
  \end{tabular}
  \end{center}
    $$H_{4}(\text{aabaabaaabaaabaa}) = \frac{1}{16} (2 H_{0}(\text{aa}) + 4 H_{0}(\text{aaaa}) + 3 H_{0}(\text{baa}) + 2 H_{0}(\text{bb}) + 1 H_{0}(\text{a}))$$
  \caption{$H_{4}(\text{aabaabaaabaaabaa})$}
\end{figure}

\section{PPM}


\chapter{Context methods II}

\section{BWC}

Burrows-Wheeler Compression (BWC) is a method based on Burrows-Wheeler Transformation (BTW), Move-to-Front usually followed by Huffman Coding.

\begin{figure*}
  \begin{center}
  \begin{tabular}{c}
    \\
    \hline
    rotations \\
    \hline
    mississippi \\
    ississippim \\
    ssissippimi \\
    sissippimis \\
    issippimiss \\
    ssippimissi \\
    sippimissis \\
    ippimississ \\
    ppimississi \\
    pimississip \\
    imississipp \\
    \hline
    \\
    \\
  \end{tabular}
  \hspace*{0.5cm}
  \begin{tabular}{c|c}
    \multicolumn{2}{c}{BWT} \\
    \hline
    i & sorted rotations \\
    \hline
    0 & imississipp \\
    1 & ippimississ \\
    2 & issippimiss \\
    3 & ississippim \\
    4 & mississippi \\
    5 & pimississip \\
    6 & ppimississi \\
    7 & sippimissis \\
    8 & sissippimis \\
    9 & ssippimissi \\
    10 & ssissippimi \\
    \hline
    \multicolumn{2}{c}{} \\
    \multicolumn{2}{c}{$4, \text{pssmipissii}$} \\
  \end{tabular}
  \hspace*{0.5cm}
  \begin{tabular}{c|c|c}
    \multicolumn{3}{c}{Move-to-Front} \\
    \hline
    input & alphabet & output \\
    \hline
    p & imps & 2 \\
    s & pims & 3 \\
    s & spim & 0 \\
    m & spim & 3 \\
    i & mspi & 3 \\
    p & imsp & 3 \\
    i & pims & 1 \\
    s & ipms & 3 \\
    s & sipm & 0 \\
    i & sipm & 1 \\
    i & ispm & 0 \\
    \hline
    \multicolumn{3}{c}{} \\
    \multicolumn{3}{c}{$4, (2,3,0,3,3,3,1,3,0,1,0)$} \\
  \end{tabular}
  \end{center}
  \caption{$BWC(\text{mississippi})$}
\end{figure*}

\begin{figure*}
  \begin{center}
  \begin{tabular}{c|c|c}
    input & alphabet & output \\
    \hline
    2 & imps & p \\
    3 & pims & s \\
    0 & spim & s \\
    3 & spim & m \\
    3 & mspi & i \\
    3 & imsp & p \\
    1 & pims & i \\
    3 & ipms & s \\
    0 & sipm & s \\
    1 & sipm & i \\
    0 & ispm & i \\
    \multicolumn{3}{c}{} \\
    \multicolumn{3}{c}{$4, \text{pssmipissii}$} \\
  \end{tabular}
  \hspace*{0.5cm}
  \begin{tabular}{c|c|c|c}
    i & sorted & input & map \\
    \hline
    0 & i & p & 5 \\
    1 & i & s & 7 \\
    2 & i & s & 8 \\
    3 & i & m & 4 \\
    4 & mississippi & i & 0 \\
    5 & p & p & 6 \\
    6 & p & i & 1 \\
    7 & s & s & 9 \\
    8 & s & s & 10 \\
    9 & s & i & 2 \\
    10 & s & i & 3 \\
    \multicolumn{4}{c}{} \\
    \multicolumn{4}{c}{mississippi} \\
  \end{tabular}
  \end{center}
  \caption{$BWC^{R}(4, (2,3,0,3,3,3,1,3,0,1,0))$}
\end{figure*}

\section{DCA}

\section{ACB}


\clearpage
\chapter*{Conslusion}

\newthought{This materials} are are still work in progress.

I thank you very much for reading them, using them and helping me improve them by your input.

To be continued\ldots

\raggedleft \emph{Ing. Radomír Polách}

%\chapter{The Design of Tufte's Books}
%\label{ch:tufte-design}
%
%\newthought{The pages} of a book are usually divided into three major
%sections: the front matter (also called preliminary matter or prelim), the
%main matter (the core text of the book), and the back matter (or end
%matter).
%
%\newthought{The front matter} of a book refers to all of the material that
%comes before the main text.  The following table from shows a list of
%material that appears in the front matter of \VDQI, \EI, \VE, and \BE
%along with its page number.  Page numbers that appear in parentheses refer
%to folios that do not have a printed page number (but they are still
%counted in the page number sequence).
%
%\bigskip
%\begin{minipage}{\textwidth}
%\begin{center}
%\begin{tabular}{lcccc}
%\toprule
% & \multicolumn{4}{c}{Books} \\
%%\cmidrule(l){2-5} 
%Page content & \vdqi & \ei & \ve & \be \\
%\midrule
%Blank half title page & \hangp{1} & \hangp{1} & \hangp{1} & \hangp{1} \\
%Frontispiece\footnotemark{}
%  & \hangp{2} & \hangp{2} & \hangp{2} & \hangp{2} \\
%Full title page & \hangp{3} & \hangp{3} & \hangp{3} & \hangp{3} \\
%Copyright page & \hangp{4} & \hangp{4} & \hangp{4} & \hangp{4} \\
%Contents & \hangp{5} & \hangp{5} & \hangp{5} & \hangp{5} \\
%%Blank & -- & \hangp{6} & \hangp{6} & \hangp{6} \\
%Dedication & \hangp{6} & \hangp{7} & \hangp{7} & 7 \\
%%Blank & -- & \hangp{8} & -- & \hangp{8} \\
%Epigraph & -- & -- & \hangp{8} & -- \\
%Introduction & \hangp{7} & \hangp{9} & \hangp{9} & 9 \\
%\bottomrule
%\end{tabular}
%\end{center}
%\end{minipage}
%\vspace{-7\baselineskip}\footnotetext{The contents of this page vary from book to book.  In
%  \vdqi this page is blank; in \ei and \ve this page holds a frontispiece;
%  and in \be this page contains three epigraphs.}
%\vspace{7\baselineskip}
%
%\bigskip
%The design of the front matter in Tufte's books varies slightly from the
%traditional design of front matter.  First, the pages in front matter are
%traditionally numbered with lowercase roman numerals (\eg, i, ii, iii,
%iv,~\ldots).  Second, the front matter page numbering sequence is usually
%separate from the main matter page numbering.  That is, the page numbers
%restart at 1 when the main matter begins.  In contrast, Tufte has
%enumerated his pages with arabic numerals that share the same page counting
%sequence as the main matter.  
%
%There are also some variations in design across Tufte's four books.  The
%page opposite the full title page (labeled ``frontispiece'' in the above
%table) has different content in each of the books.  In \VDQI, this page is
%blank; in \EI and \VE, this page holds a frontispiece; and in \BE, this
%page contains three epigraphs.
%
%The dedication appears on page~6 in \vdqi (opposite the introduction), and
%is placed on its own spread in the other books.  In \ve, an epigraph shares
%the spread with the opening page of the introduction.
%
%None of the page numbers (folios) of the front matter are expressed except in
%\be, where the folios start to appear on the dedication page.
%
%\newthought{The full title page} of each of the books varies slightly in
%design.  In all the books, the author's name appears at the top of the
%page, the title it set just above the center line, and the publisher is
%printed along the bottom margin.  Some of the differences are outlined in
%the following table.
%
%\bigskip
%\begin{center}
%\footnotesize
%\begin{tabular}{lllll}
%\toprule
%Feature & \vdqi & \ei & \ve & \be \\
%\midrule
%Author & & & & \\
%\quad Typeface & serif   & serif   & serif   & sans serif \\
%\quad Style    & italics & italics & italics & upright, caps \\
%\quad Size     & 24 pt   & 20 pt   & 20 pt   & 20 pt \\
%\addlinespace
%Title & & & & \\
%\quad Typeface & serif   & serif   & serif   & sans serif \\
%\quad Style    & upright & italics & upright & upright, caps \\
%\quad Size     & 36 pt   & 48 pt   & 48 pt   & 36 pt \\
%\addlinespace
%Subtitle & & & & \\
%\quad Typeface & \na     & \na     & serif   & \na \\
%\quad Style    & \na     & \na     & upright & \na \\
%\quad Size     & \na     & \na     & 20 pt   & \na \\
%\addlinespace
%Edition & & & & \\
%\quad Typeface & sans serif    & \na  & \na  & \na \\
%\quad Style    & upright, caps & \na  & \na  & \na \\
%\quad Size     & 14 pt         & \na  & \na  & \na \\
%\addlinespace
%Publisher & & & & \\
%\quad Typeface & serif   & serif   & serif   & sans serif \\
%\quad Style    & italics & italics & italics & upright, caps \\
%\quad Size     & 14 pt   & 14 pt   & 14 pt   & 14 pt \\
%\bottomrule
%\end{tabular}
%\end{center}
%
%\begin{figure*}[p]
%\fbox{\includegraphics[width=0.45\linewidth]{graphics/vdqi-title.pdf}}
%\hfill
%\fbox{\includegraphics[width=0.45\linewidth]{graphics/ei-title.pdf}}
%\\\vspace{\baselineskip}
%\fbox{\includegraphics[width=0.45\linewidth]{graphics/ve-title.pdf}}
%\hfill
%\fbox{\includegraphics[width=0.45\linewidth]{graphics/be-title.pdf}}
%\end{figure*}
%
%\newthought{The tables of contents} in Tufte's books give us our first
%glimpse of the structure of the main matter.  \VDQI is split into two
%parts, each containing some number of chapters.  His other three books only
%contain chapters---they're not broken into parts.
%
%\begin{figure*}[p]\index{table of contents}
%\fbox{\includegraphics[width=0.45\linewidth]{graphics/vdqi-contents.pdf}}
%\hfill
%\fbox{\includegraphics[width=0.45\linewidth]{graphics/ei-contents.pdf}}
%\\\vspace{\baselineskip}
%\fbox{\includegraphics[width=0.45\linewidth]{graphics/ve-contents.pdf}}
%\hfill
%\fbox{\includegraphics[width=0.45\linewidth]{graphics/be-contents.pdf}}
%\end{figure*}
%
%
%\section{Typefaces}\label{sec:typefaces1}\index{typefaces}
%\index{fonts|see{typefaces}}
%
%Tufte's books primarily use two typefaces: Bembo and Gill Sans.  Bembo is used
%for the headings and body text, while Gill Sans is used for the title page and
%opening epigraphs in \BE.
%
%Since neither Bembo nor Gill Sans are available in default \LaTeX{}
%installations, the \TL document classes default to using Palatino and
%Helvetica, respectively.  In addition, the Bera Mono typeface is used for
%\texttt{monospaced} type.
%
%The following font sizes are defined by the \TL classes:
%
%\begin{table}[h]\index{typefaces!sizes}
%  \footnotesize%
%  \begin{center}
%    \begin{tabular}{lccl}
%      \toprule
%      \LaTeX{} size & Font size & Leading & Used for \\
%      \midrule
%      \verb+\tiny+         &  5 &  6 & sidenote numbers \\
%      \verb+\scriptsize+   &  7 &  8 & \na \\
%      \verb+\footnotesize+ &  8 & 10 & sidenotes, captions \\
%      \verb+\small+        &  9 & 12 & quote, quotation, and verse environments \\
%      \verb+\normalsize+   & 10 & 14 & body text \\
%      \verb+\large+        & 11 & 15 & \textsc{b}-heads \\
%      \verb+\Large+        & 12 & 16 & \textsc{a}-heads, \textsc{toc} entries, author, date \\
%      \verb+\LARGE+        & 14 & 18 & handout title \\
%      \verb+\huge+         & 20 & 30 & chapter heads \\
%      \verb+\Huge+         & 24 & 36 & part titles \\
%      \bottomrule
%    \end{tabular}
%  \end{center}
%  \caption{A list of \LaTeX{} font sizes as defined by the \TL document classes.}
%  \label{tab:font-sizes}
%\end{table}
%
%\section{Headings}\label{sec:headings1}\index{headings}
%
%Tufte's books include the following heading levels: parts,
%chapters,\sidenote{Parts and chapters are defined for the \texttt{tufte-book}
%class only.}  sections, subsections, and paragraphs.  Not defined by default
%are: sub-subsections and subparagraphs.
%
%\begin{table}[h]
%  \begin{center}
%    \footnotesize%
%    \begin{tabular}{lcr}
%      \toprule
%      Heading & Style & Size \\
%      \midrule
%      Part & roman & \measure{24}{36}{40} \\
%      Chapter & italic & \measure{20}{30}{40} \\
%      Section & italic & \measure{12}{16}{26} \\
%      Subsection & italic & \measure{11}{15}{26} \\
%      Paragraph & italic & 10/14 \\
%      \bottomrule
%    \end{tabular}
%  \end{center}
%  \caption{Heading styles used in \BE.}
%  \label{tab:heading-styles}
%\end{table}
%
%\paragraph{Paragraph} Paragraph headings (as shown here) are introduced by
%italicized text and separated from the main paragraph by a bit of space.
%
%\section{Environments}
%
%The following characteristics define the various environments:
%
%
%\begin{table}[h]
%  \begin{center}
%    \footnotesize%
%    \begin{tabular}{lcl}
%      \toprule
%      Environment & Font size & Notes \\
%      \midrule
%      Body text & \measure{10}{14}{26} & \\
%      Block quote & \measure{9}{12}{24} & Block indent (left and right) by \unit[1]{pc} \\
%      Sidenotes & \measure{8}{10}{12} & Sidenote number is set inline, followed by word space \\
%      Captions & \measure{8}{10}{12} &  \\
%      \bottomrule
%    \end{tabular}
%  \end{center}
%  \caption{Environment styles used in \BE.}
%  \label{tab:environment-styles}
%\end{table}
%
%
%\chapter[On the Use of the tufte-book Document Class]{On the Use of the \texttt{tufte-book} Document Class}
%\label{ch:tufte-book}
%
%The \TL document classes define a style similar to the
%style Edward Tufte uses in his books and handouts.  Tufte's style is known
%for its extensive use of sidenotes, tight integration of graphics with
%text, and well-set typography.  This document aims to be at once a
%demonstration of the features of the \TL document classes
%and a style guide to their use.
%
%\section{Page Layout}\label{sec:page-layout}
%\subsection{Headings}\label{sec:headings}\index{headings}
%This style provides \textsc{a}- and \textsc{b}-heads (that is,
%\Verb|\section| and \Verb|\subsection|), demonstrated above.
%
%If you need more than two levels of section headings, you'll have to define
%them yourself at the moment; there are no pre-defined styles for anything below
%a \Verb|\subsection|.  As Bringhurst points out in \textit{The Elements of
%Typographic Style},\cite{Bringhurst2005} you should ``use as many levels of
%headings as you need: no more, and no fewer.''
%
%The \TL classes will emit an error if you try to use
%\linebreak\Verb|\subsubsection| and smaller headings.
%
%% let's start a new thought -- a new section
%\newthought{In his later books},\cite{Tufte2006} Tufte
%starts each section with a bit of vertical space, a non-indented paragraph,
%and sets the first few words of the sentence in \textsc{small caps}.  To
%accomplish this using this style, use the \doccmddef{newthought} command:
%\begin{docspec}
%  \doccmd{newthought}\{In his later books\}, Tufte starts\ldots
%\end{docspec}
%
%
%\section{Sidenotes}\label{sec:sidenotes}
%One of the most prominent and distinctive features of this style is the
%extensive use of sidenotes.  There is a wide margin to provide ample room
%for sidenotes and small figures.  Any \doccmd{footnote}s will automatically
%be converted to sidenotes.\footnote{This is a sidenote that was entered
%using the \texttt{\textbackslash footnote} command.}  If you'd like to place ancillary
%information in the margin without the sidenote mark (the superscript
%number), you can use the \doccmd{marginnote} command.\marginnote{This is a
%margin note.  Notice that there isn't a number preceding the note, and
%there is no number in the main text where this note was written.}
%
%The specification of the \doccmddef{sidenote} command is:
%\begin{docspec}
%  \doccmd{sidenote}[\docopt{number}][\docopt{offset}]\{\docarg{Sidenote text.}\}
%\end{docspec}
%
%Both the \docopt{number} and \docopt{offset} arguments are optional.  If you
%provide a \docopt{number} argument, then that number will be used as the
%sidenote number.  It will change of the number of the current sidenote only and
%will not affect the numbering sequence of subsequent sidenotes.
%
%Sometimes a sidenote may run over the top of other text or graphics in the
%margin space.  If this happens, you can adjust the vertical position of the
%sidenote by providing a dimension in the \docopt{offset} argument.  Some
%examples of valid dimensions are:
%\begin{docspec}
%  \ttfamily 1.0in \qquad 2.54cm \qquad 254mm \qquad 6\Verb|\baselineskip|
%\end{docspec}
%If the dimension is positive it will push the sidenote down the page; if the
%dimension is negative, it will move the sidenote up the page.
%
%While both the \docopt{number} and \docopt{offset} arguments are optional, they
%must be provided in order.  To adjust the vertical position of the sidenote
%while leaving the sidenote number alone, use the following syntax:
%\begin{docspec}
%  \doccmd{sidenote}[][\docopt{offset}]\{\docarg{Sidenote text.}\}
%\end{docspec}
%The empty brackets tell the \Verb|\sidenote| command to use the default
%sidenote number.
%
%If you \emph{only} want to change the sidenote number, however, you may
%completely omit the \docopt{offset} argument:
%\begin{docspec}
%  \doccmd{sidenote}[\docopt{number}]\{\docarg{Sidenote text.}\}
%\end{docspec}
%
%The \doccmddef{marginnote} command has a similar \docarg{offset} argument:
%\begin{docspec}
%  \doccmd{marginnote}[\docopt{offset}]\{\docarg{Margin note text.}\}
%\end{docspec}
%
%\section{References}
%References are placed alongside their citations as sidenotes,
%as well.  This can be accomplished using the normal \doccmddef{cite}
%command.\sidenote{The first paragraph of this document includes a citation.}
%
%The complete list of references may also be printed automatically by using
%the \doccmddef{bibliography} command.  (See the end of this document for an
%example.)  If you do not want to print a bibliography at the end of your
%document, use the \doccmddef{nobibliography} command in its place.  
%
%To enter multiple citations at one location,\cite[-3\baselineskip]{Tufte2006,Tufte1990} you can
%provide a list of keys separated by commas and the same optional vertical
%offset argument: \Verb|\cite{Tufte2006,Tufte1990}|.  
%\begin{docspec}
%  \doccmd{cite}[\docopt{offset}]\{\docarg{bibkey1,bibkey2,\ldots}\}
%\end{docspec}
%
%\section{Figures and Tables}\label{sec:figures-and-tables}
%Images and graphics play an integral role in Tufte's work.
%In addition to the standard \docenvdef{figure} and \docenvdef{tabular} environments,
%this style provides special figure and table environments for full-width
%floats.
%
%Full page--width figures and tables may be placed in \docenvdef{figure*} or
%\docenvdef{table*} environments.  To place figures or tables in the margin,
%use the \docenvdef{marginfigure} or \docenvdef{margintable} environments as follows
%(see figure~\ref{fig:marginfig}):
%
%\begin{marginfigure}%
%  \includegraphics[width=\linewidth]{helix}
%  \caption{This is a margin figure.  The helix is defined by 
%    $x = \cos(2\pi z)$, $y = \sin(2\pi z)$, and $z = [0, 2.7]$.  The figure was
%    drawn using Asymptote (\url{http://asymptote.sf.net/}).}
%  \label{fig:marginfig}
%\end{marginfigure}
%
%\begin{docspec}
%\textbackslash begin\{marginfigure\}\\
%  \qquad\textbackslash includegraphics\{helix\}\\
%  \qquad\textbackslash caption\{This is a margin figure.\}\\
%  \qquad\textbackslash label\{fig:marginfig\}\\
%\textbackslash end\{marginfigure\}\\
%\end{docspec}
%
%The \docenv{marginfigure} and \docenv{margintable} environments accept an optional parameter \docopt{offset} that adjusts the vertical position of the figure or table.  See the ``\nameref{sec:sidenotes}'' section above for examples.  The specifications are:
%\begin{docspec}
%  \textbackslash{begin\{marginfigure\}[\docopt{offset}]}\\
%  \qquad\ldots\\
%  \textbackslash{end\{marginfigure\}}\\
%  \mbox{}\\
%  \textbackslash{begin\{margintable\}[\docopt{offset}]}\\
%  \qquad\ldots\\
%  \textbackslash{end\{margintable\}}\\
%\end{docspec}
%
%Figure~\ref{fig:fullfig} is an example of the \docenv{figure*}
%environment and figure~\ref{fig:textfig} is an example of the normal
%\docenv{figure} environment.
%
%\begin{figure*}[h]
%  \includegraphics[width=\linewidth]{sine.pdf}%
%  \caption{This graph shows $y = \sin x$ from about $x = [-10, 10]$.
%  \emph{Notice that this figure takes up the full page width.}}%
%  \label{fig:fullfig}%
%\end{figure*}
%
%\begin{figure}
%  \includegraphics{hilbertcurves.pdf}
%%  \checkparity This is an \pageparity\ page.%
%  \caption[Hilbert curves of various degrees $n$.][6pt]{Hilbert curves of various degrees $n$. \emph{Notice that this figure only takes up the main textblock width.}}
%  \label{fig:textfig}
%  %\zsavepos{pos:textfig}
%\end{figure}
%
%As with sidenotes and marginnotes, a caption may sometimes require vertical
%adjustment. The \doccmddef{caption} command now takes a second optional
%argument that enables you to do this by providing a dimension \docopt{offset}.
%You may specify the caption in any one of the following forms:
%\begin{docspec}
%  \doccmd{caption}\{\docarg{long caption}\}\\
%  \doccmd{caption}[\docarg{short caption}]\{\docarg{long caption}\}\\
%  \doccmd{caption}[][\docopt{offset}]\{\docarg{long caption}\}\\
%  \doccmd{caption}[\docarg{short caption}][\docopt{offset}]%
%                  \{\docarg{long caption}\}
%\end{docspec}
%A positive \docopt{offset} will push the caption down the page. The short
%caption, if provided, is what appears in the list of figures/tables, otherwise
%the ``long'' caption appears there. Note that although the arguments
%\docopt{short caption} and \docopt{offset} are both optional, they must be
%provided in order. Thus, to specify an \docopt{offset} without specifying a
%\docopt{short caption}, you must include the first set of empty brackets
%\Verb|[]|, which tell \doccmd{caption} to use the default ``long'' caption. As
%an example, the caption to figure~\ref{fig:textfig} above was given in the form
%\begin{docspec}
%  \doccmd{caption}[Hilbert curves...][6pt]\{Hilbert curves...\}
%\end{docspec}
%
%Table~\ref{tab:normaltab} shows table created with the \docpkg{booktabs}
%package.  Notice the lack of vertical rules---they serve only to clutter
%the table's data.
%
%\begin{table}[ht]
%  \centering
%  \fontfamily{ppl}\selectfont
%  \begin{tabular}{ll}
%    \toprule
%    Margin & Length \\
%    \midrule
%    Paper width & \unit[8\nicefrac{1}{2}]{inches} \\
%    Paper height & \unit[11]{inches} \\
%    Textblock width & \unit[6\nicefrac{1}{2}]{inches} \\
%    Textblock/sidenote gutter & \unit[\nicefrac{3}{8}]{inches} \\
%    Sidenote width & \unit[2]{inches} \\
%    \bottomrule
%  \end{tabular}
%  \caption{Here are the dimensions of the various margins used in the Tufte-handout class.}
%  \label{tab:normaltab}
%  %\zsavepos{pos:normaltab}
%\end{table}
%
%\newthought{Occasionally} \LaTeX{} will generate an error message:\label{err:too-many-floats}
%\begin{docspec}
%  Error: Too many unprocessed floats
%\end{docspec}
%\LaTeX{} tries to place floats in the best position on the page.  Until it's
%finished composing the page, however, it won't know where those positions are.
%If you have a lot of floats on a page (including sidenotes, margin notes,
%figures, tables, etc.), \LaTeX{} may run out of ``slots'' to keep track of them
%and will generate the above error.
%
%\LaTeX{} initially allocates 18 slots for storing floats.  To work around this
%limitation, the \TL document classes provide a \doccmddef{morefloats} command
%that will reserve more slots.
%
%The first time \doccmd{morefloats} is called, it allocates an additional 34
%slots.  The second time \doccmd{morefloats} is called, it allocates another 26
%slots.
%
%The \doccmd{morefloats} command may only be used two times.  Calling it a
%third time will generate an error message.  (This is because we can't safely
%allocate many more floats or \LaTeX{} will run out of memory.)
%
%If, after using the \doccmd{morefloats} command twice, you continue to get the
%\texttt{Too many unprocessed floats} error, there are a couple things you can
%do.
%
%The \doccmddef{FloatBarrier} command will immediately process all the floats
%before typesetting more material.  Since \doccmd{FloatBarrier} will start a new
%paragraph, you should place this command at the beginning or end of a
%paragraph.
%
%The \doccmddef{clearpage} command will also process the floats before
%continuing, but instead of starting a new paragraph, it will start a new page.
%
%You can also try moving your floats around a bit: move a figure or table to the
%next page or reduce the number of sidenotes.  (Each sidenote actually uses
%\emph{two} slots.)
%
%After the floats have placed, \LaTeX{} will mark those slots as unused so they
%are available for the next page to be composed.
%
%\section{Captions}
%You may notice that the captions are sometimes misaligned.
%Due to the way \LaTeX's float mechanism works, we can't know for sure where it
%decided to put a float. Therefore, the \TL document classes provide commands to
%override the caption position.
%
%\paragraph{Vertical alignment} To override the vertical alignment, use the
%\doccmd{setfloatalignment} command inside the float environment.  For
%example:
%
%\begin{fullwidth}
%\begin{docspec}
%  \textbackslash begin\{figure\}[btp]\\
%  \qquad \textbackslash includegraphics\{sinewave\}\\
%  \qquad \textbackslash caption\{This is an example of a sine wave.\}\\
%  \qquad \textbackslash label\{fig:sinewave\}\\
%  \qquad \hlred{\textbackslash setfloatalignment\{b\}\% forces caption to be bottom-aligned}\\
%  \textbackslash end\{figure\}
%\end{docspec}
%\end{fullwidth}
%
%\noindent The syntax of the \doccmddef{setfloatalignment} command is:
%
%\begin{docspec}
%  \doccmd{setfloatalignment}\{\docopt{pos}\}
%\end{docspec}
%
%\noindent where \docopt{pos} can be either \texttt{b} for bottom-aligned
%captions, or \texttt{t} for top-aligned captions.
%
%\paragraph{Horizontal alignment}\label{par:overriding-horizontal}
%To override the horizontal alignment, use either the \doccmd{forceversofloat}
%or the \doccmd{forcerectofloat} command inside of the float environment.  For
%example:
%
%\begin{fullwidth}
%\begin{docspec}
%  \textbackslash begin\{figure\}[btp]\\
%  \qquad \textbackslash includegraphics\{sinewave\}\\
%  \qquad \textbackslash caption\{This is an example of a sine wave.\}\\
%  \qquad \textbackslash label\{fig:sinewave\}\\
%  \qquad \hlred{\textbackslash forceversofloat\% forces caption to be set to the left of the float}\\
%  \textbackslash end\{figure\}
%\end{docspec}
%\end{fullwidth}
%
%The \doccmddef{forceversofloat} command causes the algorithm to assume the
%float has been placed on a verso page---that is, a page on the left side of a
%two-page spread.  Conversely, the \doccmddef{forcerectofloat} command causes
%the algorithm to assume the float has been placed on a recto page---that is, a
%page on the right side of a two-page spread.
%
%
%\section{Full-width text blocks}
%
%In addition to the new float types, there is a \docenvdef{fullwidth}
%environment that stretches across the main text block and the sidenotes
%area.
%
%\begin{Verbatim}
%\begin{fullwidth}
%Lorem ipsum dolor sit amet...
%\end{fullwidth}
%\end{Verbatim}
%
%\begin{fullwidth}
%\small\itshape\lipsum[1]
%\end{fullwidth}
%
%\section{Typography}\label{sec:typography}
%
%\subsection{Typefaces}\label{sec:typefaces}\index{typefaces}
%If the Palatino, \textsf{Helvetica}, and \texttt{Bera Mono} typefaces are installed, this style
%will use them automatically.  Otherwise, we'll fall back on the Computer Modern
%typefaces.
%
%\subsection{Letterspacing}\label{sec:letterspacing}
%This document class includes two new commands and some improvements on
%existing commands for letterspacing.
%
%When setting strings of \allcaps{ALL CAPS} or \smallcaps{small caps}, the
%letter\-spacing---that is, the spacing between the letters---should be
%increased slightly.\cite{Bringhurst2005}  The \doccmddef{allcaps} command has proper letterspacing for
%strings of \allcaps{FULL CAPITAL LETTERS}, and the \doccmddef{smallcaps} command
%has letterspacing for \smallcaps{small capital letters}.  These commands
%will also automatically convert the case of the text to upper- or
%lowercase, respectively.
%
%The \doccmddef{textsc} command has also been redefined to include
%letterspacing.  The case of the \doccmd{textsc} argument is left as is,
%however.  This allows one to use both uppercase and lowercase letters:
%\textsc{The Initial Letters Of The Words In This Sentence Are Capitalized.}
%
%
%
%\section{Document Class Options}\label{sec:options}
%
%\index{class options|(}
%The \doccls{tufte-book} class is based on the \LaTeX\ \doccls{book}
%document class.  Therefore, you can pass any of the typical book
%options.  There are a few options that are specific to the
%\doccls{tufte-book} document class, however.
%
%The \docclsoptdef{a4paper} option will set the paper size to \smallcaps{A4} instead of
%the default \smallcaps{US} letter size.
%
%The \docclsoptdef{sfsidenotes} option will set the sidenotes and title block in a 
%\textsf{sans serif} typeface instead of the default roman.
%
%The \docclsoptdef{twoside} option will modify the running heads so that the page
%number is printed on the outside edge (as opposed to always printing the page
%number on the right-side edge in \docclsoptdef{oneside} mode).  
%
%The \docclsoptdef{symmetric} option typesets the sidenotes on the outside edge of
%the page.  This is how books are traditionally printed, but is contrary to
%Tufte's book design which sets the sidenotes on the right side of the page.
%This option implicitly sets the \docclsopt{twoside} option.
%
%The \docclsoptdef{justified} option sets all the text fully justified (flush left
%and right).  The default is to set the text ragged right.  
%The body text of Tufte's books are set ragged right.  This prevents
%needless hyphenation and makes it easier to read the text in the slightly
%narrower column.
%
%The \docclsoptdef{bidi} option loads the \docpkg{bidi} package which is used with
%\tXeLaTeX\ to typeset bi-directional text.  Since the \docpkg{bidi}
%package needs to be loaded before the sidenotes and cite commands are defined,
%it can't be loaded in the document preamble.
%
%The \docclsoptdef{debug} option causes the \TL classes to output debug
%information to the log file which is useful in troubleshooting bugs.  It will
%also cause the graphics to be replaced by outlines.
%
%The \docclsoptdef{nofonts} option prevents the \TL classes from
%automatically loading the Palatino and Helvetica typefaces.  You should use
%this option if you wish to load your own fonts.  If you're using \tXeLaTeX, this
%option is implied (\ie, the Palatino and Helvetica fonts aren't loaded if you
%use \tXeLaTeX).  
%
%The \docclsoptdef{nols} option inhibits the letterspacing code.  The \TL\
%classes try to load the appropriate letterspacing package (either pdf\TeX's
%\docpkg{letterspace} package or the \docpkg{soul} package).  If you're using
%\tXeLaTeX\ with \docpkg{fontenc}, however, you should configure your own
%letterspacing.  
%
%The \docclsoptdef{notitlepage} option causes \doccmd{maketitle} to generate a title
%block instead of a title page.  The \doccls{book} class defaults to a title
%page and the \doccls{handout} class defaults to the title block.  There is an
%analogous \docclsoptdef{titlepage} option that forces \doccmd{maketitle} to
%generate a full title page instead of the title block.
%
%The \docclsoptdef{notoc} option suppresses \TL's custom table of contents
%(\textsc{toc}) design.  The current \textsc{toc} design only shows unnumbered
%chapter titles; it doesn't show sections or subsections.  The \docclsopt{notoc}
%option will revert to \LaTeX's \textsc{toc} design.
%
%The \docclsoptdef{nohyper} option prevents the \docpkg{hyperref} package from
%being loaded.  The default is to load the \docpkg{hyperref} package and use the
%\doccmd{title} and \doccmd{author} contents as metadata for the generated
%\textsc{pdf}.
%
%\index{class options|)}
%
%
%
%\chapter[Customizing Tufte-LaTeX]{Customizing \TL}
%\label{ch:customizing}
%
%The \TL document classes are designed to closely emulate Tufte's book
%design by default.  However, each document is different and you may encounter
%situations where the default settings are insufficient.  This chapter explores
%many of the ways you can adjust the \TL document classes to better fit
%your needs.
%
%\section{File Hooks}
%\label{sec:filehooks}
%
%\index{file hooks|(}
%If you create many documents using the \TL classes, it's easier to
%store your customizations in a separate file instead of copying them into the
%preamble of each document.  The \TL classes provide three file hooks:
%\docfilehook{tufte-common-local.tex}{common}, \docfilehook{tufte-book-local.tex}{book}, and
%\docfilehook{tufte-handout-local.tex}{handout}.\sloppy
%
%\begin{description}
%  \item[\docfilehook{tufte-common-local.tex}{common}]
%    If this file exists, it will be loaded by all of the \TL document
%    classes just prior to any document-class-specific code.  If your
%    customizations or code should be included in both the book and handout
%    classes, use this file hook.
%  \item[\docfilehook{tufte-book-local.tex}{book}] 
%    If this file exists, it will be loaded after all of the common and
%    book-specific code has been read.  If your customizations apply only to the
%    book class, use this file hook.
%  \item[\docfilehook{tufte-common-handout.tex}{handout}] 
%    If this file exists, it will be loaded after all of the common and
%    handout-specific code has been read.  If your customizations apply only to
%    the handout class, use this file hook.
%\end{description}
%
%\index{file hooks|)}
%
%\section{Numbered Section Headings}
%\label{sec:numbered-sections}
%\index{headings!numbered}
%
%While Tufte dispenses with numbered headings in his books, if you require them,
%they can be anabled by changing the value of the \doccounter{secnumdepth}
%counter.  From the table below, select the heading level at which numbering
%should stop and set the \doccounter{secnumdepth} counter to that value.  For
%example, if you want parts and chapters numbered, but don't want numbering for
%sections or subsections, use the command:
%\begin{docspec}
%  \doccmd{setcounter}\{secnumdepth\}\{0\}
%\end{docspec}
%
%The default \doccounter{secnumdepth} for the \TL document classes is $-1$.
%
%\begin{table}
%  \footnotesize
%  \begin{center}
%    \begin{tabular}{lr}
%      \toprule
%      Heading level & Value \\
%      \midrule
%      Part (in \doccls{tufte-book}) & $-1$ \\
%      Part (in \doccls{tufte-handout}) & $0$ \\
%      Chapter (only in \doccls{tufte-book}) & $0$ \\
%      Section & $1$ \\
%      Subsection & $2$ \\
%      Subsubsection & $3$ \\
%      Paragraph & $4$ \\
%      Subparagraph & $5$ \\
%      \bottomrule
%    \end{tabular}
%  \end{center}
%  \caption{Heading levels used with the \texttt{secnumdepth} counter.}
%\end{table}
%
%\section{Changing the Paper Size}
%\label{sec:paper-size}
%
%The \TL classes currently only provide three paper sizes: \textsc{a4},
%\textsc{b5}, and \textsc{us} letter.  To specify a different paper size (and/or
%margins), use the \doccmd[geometry]{geometry} command in the preamble of your
%document (or one of the file hooks).  The full documentation of the
%\doccmd{geometry} command may be found in the \docpkg{geometry} package
%documentation.\cite{pkg-geometry}
%
%
%\section{Customizing Marginal Material}
%\label{sec:marginal-material}
%
%Marginal material includes sidenotes, citations, margin notes, and captions.
%Normally, the justification of the marginal material follows the justification
%of the body text.  If you specify the \docclsopt{justified} document class
%option, all of the margin material will be fully justified as well.  If you
%don't specify the \docclsopt{justified} option, then the marginal material will
%be set ragged right.
%
%You can set the justification of the marginal material separately from the body
%text using the following document class options: \docclsopt{sidenote},
%\docclsopt{marginnote}, \docclsopt{caption}, \docclsopt{citation}, and
%\docclsopt{marginals}.  Each option refers to its obviously corresponding
%marginal material type.  The \docclsopt{marginals} option simultaneously sets
%the justification on all four marginal material types.
%
%Each of the document class options takes one of five justification types:
%\begin{description}
%  \item[\docclsopt{justified}] Fully justifies the text (sets it flush left and
%    right).
%  \item[\docclsopt{raggedleft}] Sets the text ragged left, regardless of which
%    page it falls on.
%  \item[\docclsopt{raggedright}] Sets the text ragged right, regardless of
%    which page it falls on.
%  \item[\doccls{raggedouter}] Sets the text ragged left if it falls on the
%    left-hand (verso) page of the spread and otherwise sets it ragged right.
%    This is useful in conjunction with the \docclsopt{symmetric} document class
%    option.
%  \item[\docclsopt{auto}] If the \docclsopt{justified} document class option
%    was specified, then set the text fully justified; otherwise the text is set
%    ragged right.  This is the default justification option if one is not
%    explicitly specified.
%\end{description}
%
%\noindent For example, 
%\begin{docspec}
%  \doccmdnoindex{documentclass}[symmetric,justified,marginals=raggedouter]\{tufte-book\}
%\end{docspec}
%will set the body text of the document to be fully justified and all of the
%margin material (sidenotes, margin notes, captions, and citations) to be flush
%against the body text with ragged outer edges.
%
%\newthought{The font and style} of the marginal material may also be modified using the following commands:
%
%\begin{docspec}
%  \doccmd{setsidenotefont}\{\docopt{font commands}\}\\
%  \doccmd{setcaptionfont}\{\docopt{font commands}\}\\
%  \doccmd{setmarginnotefont}\{\docopt{font commands}\}\\
%  \doccmd{setcitationfont}\{\docopt{font commands}\}
%\end{docspec}
%
%The \doccmddef{setsidenotefont} sets the font and style for sidenotes, the
%\doccmddef{setcaptionfont} for captions, the \doccmddef{setmarginnotefont} for
%margin notes, and the \doccmddef{setcitationfont} for citations.  The
%\docopt{font commands} can contain font size changes (e.g.,
%\doccmdnoindex{footnotesize}, \doccmdnoindex{Huge}, etc.), font style changes (e.g.,
%\doccmdnoindex{sffamily}, \doccmdnoindex{ttfamily}, \doccmdnoindex{itshape}, etc.), color changes (e.g.,
%\doccmdnoindex{color}\texttt{\{blue\}}), and many other adjustments.
%
%If, for example, you wanted the captions to be set in italic sans serif, you could use:
%\begin{docspec}
%  \doccmd{setcaptionfont}\{\doccmdnoindex{itshape}\doccmdnoindex{sffamily}\}
%\end{docspec}
%
%\chapter{Compatibility Issues}
%\label{ch:compatibility}
%
%When switching an existing document from one document class to a \TL document class, a few changes to the document may have to be made.
%
%\section{Converting from \doccls{article} to \doccls{tufte-handout}}
%
%The following \doccls{article} class options are unsupported: \docclsopt{10pt}, \docclsopt{11pt}, \docclsopt{12pt}, \docclsopt{a5paper}, \docclsopt{b5paper}, \docclsopt{executivepaper}, \docclsopt{legalpaper}, \docclsopt{landscape}, \docclsopt{onecolumn}, and \doccls{twocolumn}.
%
%The following headings are not supported: \doccmd{subsubsection} and \doccmd{subparagraph}.
%
%\section{Converting from \doccls{book} to \doccls{tufte-book}}
%
%The following \doccls{report} class options are unsupported: \docclsopt{10pt}, \docclsopt{11pt}, \docclsopt{12pt}, \docclsopt{a5paper}, \docclsopt{b5paper}, \docclsopt{executivepaper}, \docclsopt{legalpaper}, \docclsopt{landscape}, \docclsopt{onecolumn}, and \doccls{twocolumn}.
%
%The following headings are not supported: \doccmd{subsubsection} and \doccmd{subparagraph}.
%
%
%
%\chapter{Troubleshooting and Support}
%\label{ch:troubleshooting}
%
%\section{\TL Website}\label{sec:website}
%The website for the \TL packages is located at
%\url{https://github.com/Tufte-LaTeX/tufte-latex}.  On our website, you'll find
%links to our \smallcaps{svn} repository, mailing lists, bug tracker, and documentation.
%
%\section{\TL Mailing Lists}\label{sec:mailing-lists}
%There are two mailing lists for the \TL project:
%
%\paragraph{Discussion list}
%The \texttt{tufte-latex} discussion list is for asking questions, getting
%assistance with problems, and help with troubleshooting.  Release announcements
%are also posted to this list.  You can subscribe to the \texttt{tufte-latex}
%discussion list at \url{http://groups.google.com/group/tufte-latex}.
%
%\paragraph{Commits list}
%The \texttt{tufte-latex-commits} list is a read-only mailing list.  A message
%is sent to the list any time the \TL code has been updated.  If you'd like to
%keep up with the latest code developments, you may subscribe to this list.  You
%can subscribe to the \texttt{tufte-latex-commits} mailing list at
%\url{http://groups.google.com/group/tufte-latex-commits}.
%
%\section{Getting Help}\label{sec:getting-help}
%If you've encountered a problem with one of the \TL document classes, have a
%question, or would like to report a bug, please send an email to our
%mailing list or visit our website.
%
%To help us troubleshoot the problem more quickly, please try to compile your
%document using the \docclsopt{debug} class option and send the generated
%\texttt{.log} file to the mailing list with a brief description of the problem.
%
%
%
%\section{Errors, Warnings, and Informational Messages}\label{sec:tl-messages}
%The following is a list of all of the errors, warnings, and other messages generated by the \TL classes and a brief description of their meanings.
%\index{error messages}\index{warning messages}\index{debug messages}
%
%% Errors
%\docmsg{Error: \doccmd{subparagraph} is undefined by this class.}{%
%The \doccmd{subparagraph} command is not defined in the \TL document classes.
%If you'd like to use the \doccmd{subparagraph} command, you'll need to redefine
%it yourself.  See the ``Headings'' section on page~\pageref{sec:headings} for a
%description of the heading styles available in the \TL document classes.}
%
%\docmsg{Error: \doccmd{subsubsection} is undefined by this class.}{%
%The \doccmd{subsubsection} command is not defined in the \TL document classes.
%If you'd like to use the \doccmd{subsubsection} command, you'll need to
%redefine it yourself.  See the ``Headings'' section on
%page~\pageref{sec:headings} for a description of the heading styles available
%in the \TL document classes.}
%
%\docmsg{Error: You may only call \doccmd{morefloats} twice. See the\par\noindent\ \ \ \ \ \ \ \ Tufte-LaTeX documentation for other workarounds.}{%
%\LaTeX{} allocates 18 slots for storing floats.  The first time
%\doccmd{morefloats} is called, it allocates an additional 34 slots.  The second
%time \doccmd{morefloats} is called, it allocates another 26 slots.
%
%The \doccmd{morefloats} command may only be called two times.  Calling it a
%third time will generate this error message.  See
%page~\pageref{err:too-many-floats} for more information.}
%
%% Warnings
%\docmsg{Warning: Option `\docopt{class option}' is not supported -{}- ignoring option.}{%
%This warning appears when you've tried to use \docopt{class option} with a \TL
%document class, but \docopt{class option} isn't supported by the \TL document
%class.  In this situation, \docopt{class option} is ignored.}
%
%% Info / Debug messages
%\docmsg{Info: The `\docclsopt{symmetric}' option implies `\docclsopt{twoside}'}{%
%You specified the \docclsopt{symmetric} document class option.  This option automatically forces the \docclsopt{twoside} option as well.  See page~\pageref{clsopt:symmetric} for more information on the \docclsopt{symmetric} class option.}
%
%
%\section{Package Dependencies}\label{sec:dependencies}
%The following is a list of packages that the \TL document
%classes rely upon.  Packages marked with an asterisk are optional.
%\begin{multicols}{2}
%\begin{itemize}
%  \item xifthen
%  \item ifpdf*
%  \item ifxetex*
%  \item hyperref
%  \item geometry
%  \item ragged2e
%  \item chngpage \emph{or} changepage
%  \item paralist
%  \item textcase
%  \item soul*
%  \item letterspace*
%  \item setspace
%  \item natbib \emph{and} bibentry
%  \item optparams
%  \item placeins
%  \item mathpazo*
%  \item helvet*
%  \item fontenc
%  \item beramono*
%  \item fancyhdr
%  \item xcolor
%  \item textcomp
%  \item titlesec
%  \item titletoc
%\end{itemize}
%\end{multicols}




%%
% The back matter contains appendices, bibliographies, indices, glossaries, etc.







\backmatter

\bibliography{sample-handout}
\bibliographystyle{plainnat}


\printindex

\end{document}

