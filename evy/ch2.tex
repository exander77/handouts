\chapter{Suffix structures}

\section{Suffix array}

There are several algoirthms for the Suffix Array construction. The most prominent is DC3 (Difference Cover, size 3) and which is a $O(n)$ time complexity divide-and-conquer algorithm.

For $T=abaabababbabbb$ calculate Suffix Array (\textit{SA}) and Longest Common Prefix (\textit{LCP}).

\begin{table}[h]
  \footnotesize
	\begin{tabular}{c|cccccccccccccccc}
		        $i$ 		& -1 & 0   & 1   & 2   & 3   & 4   & 5   & 6   & 7   & 8   & 9   & 10  & 11  & 12  & 13  & 14 \\ 
		\hline $T[i]$ 	& \# & $a$ & $b$ & $a$ & $a$ & $b$ & $a$ & $b$ & $a$ & $b$ & $b$ & $a$ & $b$ & $b$ & $b$ & \$ \\ 
		\hline $SA[i]$ 	& 14 & 2   & 0   & 3   & 5   & 7   & 10  & 13  & 1   & 4   & 6   & 9   & 12  & 8   & 11  & (-1) \\ 
		\hline $LCP[i]$ & 0  & 0   & 1   & 3   & 4   & 2   & 3   & 0   & 1   & 2   & 3   & 4   & 1   & 2   & 2   & 0 \\ 
	\end{tabular} 
  \caption{Suffix Array for $T=abaabababbabbb$.}
\end{table}

Suffix Array can be constructed through simple sorting by adding padding characters \# and \$, where \$ $\leq a \in \Sigma \leq$ \#:\\
$T_{padded}=$\#$abaabababbabbb\$$.

Suffixes can be easily sorted, see~\ref{sortingsuffixes}.

\begin{table}[h]
  \footnotesize
	\begin{tabular}{c|l}
		i & unsorted \\ \hline
		-1 & \#abaabababbabbb\$ \\
		0 & abaabababbabbb\$ \\
		1 & baabababbabbb\$ \\
		2 & aabababbabbb\$ \\
		3 & abababbabbb\$ \\
		4 & bababbabbb\$ \\
		5 & ababbabbb\$ \\
		6 & babbabbb\$ \\
		7 & abbabbb\$ \\
		8 & bbabbb\$ \\
		9 & babbb\$ \\
		10 & abbb\$ \\
		11 & bbb\$ \\
		12 & bb\$ \\
		13 & b\$ \\
		14 & \$ \\
	\end{tabular} 
  \hspace*{1cm}
	\begin{tabular}{c|l}
		SA[i] & sorted \\ \hline
		14 & \$ \\
		2 & aabababbabbb\$ \\
		0 & abaabababbabbb\$ \\
		3 & abababbabbb\$ \\
		5 & ababbabbb\$ \\
		7 & abbabbb\$ \\
		10 & abbb\$ \\
		13 & b\$ \\
		1 & baabababbabbb\$ \\
		4 & bababbabbb\$ \\
		6 & babbabbb\$ \\
		9 & babbb\$ \\
		12 & bb\$ \\
		8 & bbabbb\$ \\
		11 & bbb\$ \\
		-1 & \#abaabababbabbb\$ \\
	\end{tabular} 
	\label{sortingsuffixes}
	\caption{Sorting suffixes of $T_{padded}=$\#$abaabababbabbb\$$.}
\end{table}

For $P=abaa$ find all occurances using \textit{SA Simple Search}.

\begin{enumerate}
	\item \begin{itemize}
			\item $d = -1$, $f = n = 14$
		\end{itemize}
	\item \begin{itemize}
		\item $i \leftarrow (-1 + 14)/2 = 6$
		\item $L_6 = T[SA[6] .. 13] = T[13 .. 13] = b$
		\item $l \leftarrow lcp(x, L_6) = 0$
		\item $L_6[0] > x[0] \Rightarrow b > a \Rightarrow f \leftarrow i = 6$
	\end{itemize}
	\item \begin{itemize}
		\item $i \leftarrow (-1 + 6)/2 = 2$
		\item $L_2 = T[SA[2] .. 13] = T[3 .. 13] = abababbabbb$
		\item $l \leftarrow lcp(x, L_2) = 3$
		\item $L_2[3] > x[3] \Rightarrow b > a \Rightarrow f \leftarrow i = 2$
	\end{itemize}
\item \begin{itemize}
		\item $i \leftarrow (-1 + 2)/2 = 0$
		\item $L_0 = T[SA[0] .. 13] = T[2 .. 13] = aabababbabbb$
		\item $l \leftarrow lcp(x, L_0) = 1$
		\item $L_0[1] < x[1] \Rightarrow a < b \Rightarrow d \leftarrow i = 0$
	\end{itemize}
\item \begin{itemize}
		\item $i \leftarrow (0 + 2)/2 = 1$
		\item $L_1 = T[SA[1] .. 13] = T[0 .. 13] = abaabababbabbb$
		\item $l \leftarrow lcp(x, L_1) = 4$
		\item $l = m \wedge l \neq |L_1| \Rightarrow f \leftarrow i = 1$
	\end{itemize}
\item \begin{itemize}
		\item $d+1 \not< f \Rightarrow (d, f) = (0, 1)$
	\end{itemize}
\end{enumerate}

\section{Suffix trie}

For $T=ababbab$ construct Suffix Trie.

\tikzstyle{a} = [above]
\tikzstyle{c} = [circle, draw, minimum width=0.4cm]
\tikzstyle{e} = [circle, draw, minimum width=0.4cm, accepting]
\tikzstyle{level 1} = [level distance=1cm, sibling distance=2cm]
\tikzstyle{level 2} = [level distance=1cm, sibling distance=1cm]
\tikzstyle{level 3} = [level distance=1cm, sibling distance=1cm]
\tikzstyle{level 4} = [level distance=1cm, sibling distance=1cm]
\begin{tikzpicture}[grow=right]
	\node[c] {}
	child {
		node[c] {}
		child {
			node[e] {}
			child {
				node[c] {}
				child {
					node[c] {}
					child {
						node[c] {}
						child {
							node[c] {}
							child {
								node[e] {}
								edge from parent
								node[a] {$b$}
							}
							edge from parent
							node[a] {$a$}
						}
						edge from parent
						node[a] {$b$}
					}
					edge from parent
					node[a] {$b$}
				}
				edge from parent
				node[a] {$a$}
			}
			child {
				node[c] {}
				child {
					node[c] {}
					child {
						node[e] {}
						edge from parent
						node[a] {$b$}
					}
					edge from parent
					node[a] {$a$}
				}
				edge from parent
				node[a] {$b$}
			}
			edge from parent
			node[a] {$b$}
		}
		edge from parent
		node[a] {$a$}
	}
	child {
		node[e] {}
		child {
			node[c] {}
			child {
				node[e] {}
				child {
					node[c] {}
					child {
						node[c] {}
						child {
							node[e] {}
							edge from parent
							node[a] {$b$}
						}
						edge from parent
						node[a] {$a$}
					}
					edge from parent
					node[a] {$b$}
				}
				edge from parent
				node[a] {$b$}
			}
			edge from parent
			node[a] {$a$}
		}
		child {
			node[c] {}
			child {
				node[c] {}
				child {
					node[e] {}
					edge from parent
					node[a] {$b$}
				}
				edge from parent
				node[a] {$a$}
			}
			edge from parent
			node[a] {$b$}
		}
		edge from parent
		node[a] {$b$}
	};
\end{tikzpicture}


\section{Suffix tree}

For $T=ababbab$ construct Suffix Tree.

\tikzstyle{a} = [above]
\tikzstyle{c} = [circle, draw, minimum width=0.4cm]
\tikzstyle{e} = [circle, draw, minimum width=0.4cm, accepting]
\tikzstyle{level 1} = [level distance=1cm, sibling distance=2cm]
\tikzstyle{level 2} = [level distance=1cm, sibling distance=1cm]
\tikzstyle{level 3} = [level distance=1cm, sibling distance=1cm]
\tikzstyle{level 4} = [level distance=1cm, sibling distance=1cm]
\begin{tikzpicture}[grow=right]
	\node[c] {}
	child {
		node[e] {}
		child {
			node[e] {}
			edge from parent
			node[above right] {$abbab$}
		}
		child {
			node[e] {}
			edge from parent
			node[a] {$bab$}
		}
		edge from parent
		node[above right] {$ab$}
	}
	child {
		node[e] {}
		child {
			node[e] {}
			child {
				node[e] {}
				edge from parent
				node[a] {$bab$}
			}
			edge from parent
			node[above right] {$ab$}
		}
		child {
			node[e] {}
			edge from parent
			node[a] {$bab$}
		}
		edge from parent
		node[a] {$b$}
	};
\end{tikzpicture}

\section{Suffix automaton}

For $T=ababbab$ construct Suffix Automaton.

\begin{tikzpicture}[->,>=stealth',shorten >=1pt,auto]
	\tikzstyle{every state} = [circle,draw,minimum size=0.4cm]
	\node[initial,state]	(a)					{};
	\node[state]			(b) [right of=a]	{};
	\node[state,accepting]	(c) [right of=b]	{};
	\node[state]			(d) [right of=c]	{};
	\node[state]			(e) [right of=d]	{};
	\node[state]			(f) [right of=e]	{};
	\node[state]			(g) [right of=f]	{};
	\node[state,accepting]	(h) [right of=g]	{};
	\node[state,accepting]			(i) [below of=c]	{};
	\node[state]			(j) [below of=d]	{};
	\node[state,accepting]	(k) [below of=e]	{};
	\path	(a) edge node {$a$} (b)
				edge [below] node {$b$} (i)
			(b) edge node {$b$} (c)
			(c) edge node {$a$} (d)
				edge [bend left] node {$b$} (f)
			(d) edge node {$b$} (e)
			(e) edge node {$b$} (f)
			(f) edge node {$a$} (g)
			(g) edge node {$b$} (h)
			(i) edge [below] node {$b$} (f)
				edge [below] node {$a$} (j)
			(j) edge [below] node {$b$} (k)
			(k) edge [below] node {$b$} (h);
\end{tikzpicture}

\section{Compressed Suffix automaton}

For $T=ababbab$ construct Compressed Suffix Automaton.

\begin{tikzpicture}[->,>=stealth',shorten >=1pt,auto]
	\tikzstyle{every state} = [circle,draw,minimum size=0.4cm]
	\node[initial,state]	(a)					{};
	\node[state,accepting]	(c) [right of=b]	{};
	\node[state]			(f) [right of=e]	{};
	\node[state,accepting]	(h) [right of=g]	{};
	\node[state,accepting]			(i) [below of=c]	{};
	\node[state,accepting]	(k) [below of=e]	{};
	\path	(a) edge node {$ab$} (c)
				edge node {$b$} (i)
			(c) edge node {$abb$} (f)
				edge [bend left] node {$b$} (f)
			(f) edge node {$ab$} (h)
			(i) edge node {$b$} (f)
				edge node {$ab$} (k)
			(k) edge node {$b$} (h);
\end{tikzpicture}

\clearpage
\section{Suffix Automaton Construction from Suffix Trie}

For $T=ababbab$ construct Suffix Automaton from its Suffix Trie.

States of Suffix Trie are labeled as if it was an automaton.

\tikzstyle{a} = [above]
\tikzstyle{c} = [circle, draw, minimum width=0.4cm]
\tikzstyle{e} = [circle, draw, minimum width=0.4cm, accepting]
\tikzstyle{level 1} = [level distance=1cm, sibling distance=2cm]
\tikzstyle{level 2} = [level distance=1cm, sibling distance=1cm]
\tikzstyle{level 3} = [level distance=1cm, sibling distance=1cm]
\tikzstyle{level 4} = [level distance=1cm, sibling distance=1cm]
\begin{tikzpicture}[grow=right]
	\node[c] {$0$}
	child {
		node[c] {$10$}
		child {
			node[e] {$11$}
			child {
				node[c] {$15$}
				child {
					node[c] {$16$}
					child {
						node[c] {$17$}
						child {
							node[c] {$18$}
							child {
								node[e] {$19$}
								edge from parent
								node[a] {$b$}
							}
							edge from parent
							node[a] {$a$}
						}
						edge from parent
						node[a] {$b$}
					}
					edge from parent
					node[a] {$b$}
				}
				edge from parent
				node[a] {$a$}
			}
			child {
				node[c] {$12$}
				child {
					node[c] {$13$}
					child {
						node[e] {$14$}
						edge from parent
						node[a] {$b$}
					}
					edge from parent
					node[a] {$a$}
				}
				edge from parent
				node[a] {$b$}
			}
			edge from parent
			node[a] {$b$}
		}
		edge from parent
		node[a] {$a$}
	}
	child {
		node[e] {$1$}
		child {
			node[c] {$5$}
			child {
				node[e] {$6$}
				child {
					node[c] {$7$}
					child {
						node[c] {$8$}
						child {
							node[e] {$9$}
							edge from parent
							node[a] {$b$}
						}
						edge from parent
						node[a] {$a$}
					}
					edge from parent
					node[a] {$b$}
				}
				edge from parent
				node[a] {$b$}
			}
			edge from parent
			node[a] {$a$}
		}
		child {
			node[c] {$2$}
			child {
				node[c] {$3$}
				child {
					node[e] {$4$}
					edge from parent
					node[a] {$b$}
				}
				edge from parent
				node[a] {$a$}
			}
			edge from parent
			node[a] {$b$}
		}
		edge from parent
		node[a] {$b$}
	};
\end{tikzpicture}

Transition function is iteratively analyzed. In each iteration the states are put into groups based on their similarity. In the initial interation $g_0$ states are split into two groups: final states and non-final states. The groups are labeled by a first state in the group for convenience. In following in iterations $g_i, i \geq 1$ states are split into groups based on the previsous group and groups to which transions for all symbols of the alphabet lead.

\begin{table}[h]
  \footnotesize
	\begin{tabular}{|c||c|c||c||c|c||c||c|c||c||c|c||c||c|c||c||}
		\hline
		$\delta$ & a & b & $g_0$ & a & b & $g_1$ & a & b & $g_2$ & a & b & $g_3$ & a & b & $g_4$\\ \hline
		0  & 10 & 1  & 0 & 0 & 1 & 0 & 0 & 1 & 0  & 0  & 1  & 0 & 0  & 1  & 0 \\ 
		1  & 5  & 2  & 1 & 0 & 0 & 1 & 0 & 2 & 1  & 0  & 2  & 1 & 0  & 2  & 1 \\
		2  & 3  & -  & 0 & 0 & 0 & 2 & 0 & 2 & 2  & 0  & -  & 2 & 0  & -  & 2 \\
		3  & -  & 4  & 0 & 0 & 1 & 0 & 2 & 1 & 0  & -  & 4  & 3 & -  & 4  & 3 \\
		4  & -  & -  & 1 & 0 & 0 & 1 & 2 & 2 & 4  & -  & -  & 4 & -  & -  & 4 \\
		5  & -  & 6  & 0 & 0 & 1 & 0 & 2 & 1 & 0  & -  & 1  & 5 & -  & 1  & 5 \\
		6  & -  & 7  & 1 & 0 & 0 & 1 & 2 & 2 & 1  & -  & 2  & 6 & -  & 2  & 6 \\
		7  & 8  & -  & 0 & 0 & 0 & 2 & 0 & 2 & 2  & 0  & -  & 2 & 0  & -  & 2 \\
		8  & -  & 9  & 0 & 0 & 1 & 0 & 2 & 1 & 0  & -  & 4  & 3 & -  & 4  & 3 \\
		9  & -  & -  & 1 & 0 & 0 & 1 & 2 & 2 & 4  & -  & -  & 4 & -  & -  & 4 \\
		10 & -  & 11 & 0 & 0 & 1 & 0 & 2 & 1 & 0  & -  & 11 & 10& -  & 11 & 10 \\
		11 & 15 & 12 & 1 & 0 & 0 & 1 & 2 & 2 & 11 & 15 & 2  & 11& 15 & 2  & 11\\
		12 & 13 & -  & 0 & 0 & 0 & 2 & 0 & 2 & 2  & 0  & -  & 2 & 3  & -  & 2 \\
		13 & -  & 14 & 0 & 0 & 1 & 0 & 2 & 1 & 0  & -  & 4  & 3 & -  & 4  & 3 \\
		14 & -  & -  & 1 & 0 & 0 & 1 & 2 & 2 & 4  & -  & -  & 4 & -  & -  & 4 \\
		15 & -  & 16 & 0 & 0 & 0 & 2 & 2 & 2 & 15 & -  & 2  & 15& -  & 2  & 15\\
		16 & -  & 17 & 0 & 0 & 0 & 2 & 2 & 2 & 15 & -  & 15 & 16& -  & 15 & 16\\
		17 & 18 & -  & 0 & 0 & 0 & 2 & 0 & 2 & 2  & 0  & -  & 2 & 3  & -  & 2 \\
		18 & -  & 19 & 0 & 0 & 1 & 0 & 2 & 1 & 0  & -  & 4  & 3 & -  & 4  & 3 \\
		19 & -  & -  & 1 & 0 & 0 & 1 & 2 & 2 & 4  & -  & -  & 4 & -  & -  & 4 \\
		-  & -  & -  & 0 & 0 & 0 & 2 & 2 & 2 & -  & -  & -  & - & -  & -  & - \\
		\hline
	\end{tabular} 
	\caption{Minimization of Suffix Trie for string $T=ababbab$.}
\end{table}

The algorithm ends when either each state is in its own group -- the automaton in already minimal or if two consecutive iterations $g_i$ and $g_{i+1}$ split the states the same way.

States which end up in the same group are equivalent:
\begin{itemize}
	\item $2 \Leftrightarrow 7 \Leftrightarrow 12 \Leftrightarrow 17$,
	\item $3 \Leftrightarrow 8 \Leftrightarrow 13 \Leftrightarrow 18$,
	\item $4 \Leftrightarrow 9 \Leftrightarrow 14 \Leftrightarrow 19$.
\end{itemize}

The equivalent states can be merged together and minimal automaton can be constructed.

\begin{tikzpicture}[->,>=stealth',shorten >=1pt,auto]
	\tikzstyle{every state} = [circle,draw,minimum size=0.4cm]
	\node[initial,state]	(a)					{$0$};
	\node[state]			(b) [right of=a]	{$10$};
	\node[state,accepting]	(c) [right of=b]	{$11$};
	\node[state]			(d) [right of=c]	{$15$};
	\node[state]			(e) [right of=d]	{$16$};
	\node[state]			(f) [right of=e]	{$2$};
	\node[state]			(g) [right of=f]	{$3$};
	\node[state,accepting]	(h) [right of=g]	{$4$};
	\node[state,accepting]			(i) [below of=c]	{$1$};
	\node[state]			(j) [below of=d]	{$5$};
	\node[state,accepting]	(k) [below of=e]	{$6$};
	\path	(a) edge node {$a$} (b)
				edge [below] node {$b$} (i)
			(b) edge node {$b$} (c)
			(c) edge node {$a$} (d)
				edge [bend left] node {$b$} (f)
			(d) edge node {$b$} (e)
			(e) edge node {$b$} (f)
			(f) edge node {$a$} (g)
			(g) edge node {$b$} (h)
			(i) edge [below] node {$b$} (f)
				edge [below] node {$a$} (j)
			(j) edge [below] node {$b$} (k)
			(k) edge [below] node {$b$} (h);
\end{tikzpicture}
