\chapter{Coding of integers}

Coding of integers is important prerequisite to data compression. Some of these codes are unable to directly encode $0$ and none of these codes are able to directly encode negative numbers. Generally, these codes can be classified as uniquely decodable and non-uniquely decodable.

\section{Unary codes}
Encoding\sidenote{$\alpha$-code is uniquely decodable (prefix) code with variable length. It does not have the ability to encode value $0$ or negative numbers directly, usually shift of values is introduced to handle such cases, for example encoding $x+1$ or $x+k, k\geq1$.}:

\subsection{$\alpha$-code}
$$\alpha(x)=0^{x-1}1$$
\subsection{$\alpha'$-code}
This variant has $0$ and $1$ swapped.\sidenote{This variant of $\alpha$-code is primarily used in Golomb Codes.}
$$\alpha'(x)=1^{x-1}0$$
\section{Binary codes}
Encoding\sidenote{$\beta$-codes are non-uniquely decodable codes with variable length. Fixed length binary codes are also widely used, they are uniquely decodable, but alsways encode to the same number of bits and they have a maximum value. $\beta'$-code has leading $1$ removed.}:
\subsection{$\beta$-code}
$$\beta(0)=0$$
$$\beta(1)=1$$
$$\beta(2i+j)=\beta(i)\beta(j)$$
\subsection{$\beta'$-code}
$$\beta'(x) = \beta(x)-10^{|\beta(x)|-1}$$

\section{Ternary codes}
These codes introduce a special symbol $\#$ for marking end of each keyword. \sidenote{$\tau$-codes are uniquely decodable (prefix) codes with variable length.}

\noindent
Encoding:

\subsection{$\tau$-code}
$$\tau(x) = \beta(x)\#$$
\subsection{$\tau'$-code}
$$\tau'(x) = \beta'(x)\#$$

\section{Elias codes}
Encoding\sidenote{$\gamma$-codes are uniquely decodable codes with variable length. Uniquely decodable $\alpha$-code is combined with non-uniquely decodable $\beta$-code. The difference between $\gamma'$-code and $\gamma$-code is that $\gamma$-code is interleaved. $\delta$-code is similar to $\gamma'$-code, but $\alpha$-code is replaced with $\gamma$-code.}:

\subsection{$\gamma'$-code}
$$\gamma'(x)=\alpha(|\beta(x)|)\beta'(x)$$

\subsection{$\gamma$-code}
$$y=\alpha(|\beta(x)|), z=\beta'(x)$$
$$\gamma(x)=y_1,z_1,y_2,z_2,\ldots,z_{|\beta(x)|-1},y_{|\beta(x)|}$$

\subsection{$\delta$-code}
$$\gamma(x)=\gamma(|\beta(x)|)\beta'(x)$$

\noindent
The $\omega$-codes have recursive prefix. The value is encoded by $\beta$-code followed with $0$. If the encoded $\beta$-code value is longer than $k, k \geq 2$ then a prefix is recursively prolonged by the prepending the value of $\beta$-code of the length of the previous part encoded by $\beta'$-code (length of $\beta$-code $-1$). If the length of $\beta$-code is shorted then $k$ then zerozes are prepended to make it of length $k$.

\noindent
Encoding\sidenote{$\omega$-codes are uniquely decodable codes with variable length. They do not have the ability to encode value $0$ or negative numbers directly}

\subsection{$\omega$-code ($k=2$)}
$$\omega(1) = 0^{k-1} = 0$$
$$\omega(x)= \text{<prefix>}\beta(x)0$$

\subsection{$\omega'$-code ($k=3$)}
$$\omega(1) = 0^{k-1} = 00$$
$$\omega(x)= \text{<prefix>}\beta(x)0$$

\section{Ternary Comma Code}
It is a simple code based on base 3 number representation and comma (c), which gives the following encoding:

\begin{table}
\begin{tabular}{|c|c|c|c|}\hline
    0 & 1 & 2 & c\\\hline
    00 & 01 & 10 & 11\\\hline
\end{tabular}
\caption{Ternary Comma Code}
\end{table}

\section{Fibbonacci Code}
Binary number is representation where each position corresponts to $2^i, i \in \mathbb{N}_0$. These powers of $2$ can be raplaced with numbers from Fibbonacci sequence, excluding the initial $1$:
$$1, 2, 3, 5, 8, 13, 21, 34\ldots$$.

This gives us multiple representations of each number as the two consecutive numbers can always be replaced with the next one, i.e. $11$ has the same meaning as $100$, for example: 
$$1*5+1*3=1*8+0*5+0*3.$$
Because of this, each number can be represented using this Fibbonacci representation as a sequeance of $0$ and $1$, where no two consecutive $1$ are present ($11$). When such number is reversed and additional $1$ is appended, it leads to a code, where each additional $1$ marks the end of the keyword.

\section{Golomb Code}
$GC(x,m)$ is Golomb Code of $x$ in modulo $m$.

\noindent
Quotient:
$$q=\floor{x/m}.$$
Reminder:
$$r=x-qm.$$
Coeficient:
$$c=\ceil{\log_2 m}.$$

\noindent
Reminder table $R$:\\
\begin{tabular}{r l l l l}
    $2^c-m$ & values of & $r$ & $\beta$-code encoded by $c-1$ & bits,\\
    $2m-2^c$ & values of & $r+2^c-m$ & $\beta$-code encoded by $c$ & bits.
\end{tabular}

\noindent
Encoding:
$$GC(x,m)=\alpha'(q+1)R(r).$$

\subsection{Examples of reminder tables}
$R$ for $m=3$:

\begin{table}
\begin{tabular}{|c||c|c|}\hline
    0 & 1 & 2\\\hline
    0 & 10 & 11\\\hline
\end{tabular}
\caption{Golomb Code $R$ for $m=3$}
\end{table}

\noindent
$R$ for $m=4$:

\begin{table}
\begin{tabular}{||c|c|c|c|}\hline
    0 & 1 & 2 & 3\\\hline
    00 & 01 & 10 & 11\\\hline
\end{tabular}
\caption{Golomb Code $R$ for $m=4$}
\end{table}

\noindent
$R$ for $m=5$:

\begin{table}
\begin{tabular}{|c|c|c||c|c|}\hline
    0 & 1 & 2 & 3 & 4\\\hline
    00 & 01 & 10 & 110 & 111\\\hline
\end{tabular}
\caption{Golomb Code $R$ for $m=5$}
\end{table}

\noindent
$R$ for $m=6$:

\begin{table}
\begin{tabular}{|c|c||c|c|c|c|}\hline
    0 & 1 & 2 & 3 & 4 & 5\\\hline
    00 & 01 & 100 & 101 & 110 & 111\\\hline
\end{tabular}
\caption{Golomb Code $R$ for $m=6$}
\end{table}

\noindent
$R$ for $m=7$:

\begin{table}
\begin{tabular}{|c||c|c|c|c|c|c|}\hline
    0 & 1 & 2 & 3 & 4 & 5 & 6\\\hline
    00 & 010 & 011 & 100 & 101 & 110 & 111\\\hline
\end{tabular}
\caption{Golomb Code $R$ for $m=7$}
\end{table}

\noindent
$R$ for $m=8$:

\begin{table}
\begin{tabular}{||c|c|c|c|c|c|c|c|}\hline
    0 & 1 & 2 & 3 & 4 & 5 & 6 & 7\\\hline
    000 & 001 & 010 & 011 & 100 & 101 & 110 & 111\\\hline
\end{tabular}
\caption{Golomb Code $R$ for $m=8$}
\end{table}

\noindent
$R$ for $m=14$:

\begin{table*}
\begin{tabular}{|c|c||c|c|c|c|c|c|c|c|c|c|c|c|}\hline
    0 & 1 & 2 & 3 & 4 & 5 & 6 & 7 & 8 & 9 & 10 & 11 & 12 & 13\\\hline
    000 & 001 & 0100 & 0101 & 0110 & 0111 & 1000 & 1001 & 1010 & 1011 & 1100 & 1101 & 1110 & 1111\\\hline
\end{tabular}
\\
\caption{Golomb Code $R$ for $m=14$}
\end{table*}

\section{Rice Code}
$RC(x,m) = GC(x,m)$ if $m = 2^k, k \in \mathbb{N}_0$. 

%\section{Properties of codes}

%Universal, Asymptotically optimal

%\section{Examples}

%\begin{table*}
%\begin{tabular}{|c|r|r|r|r|r|r|r|r|r|r|r|r|r|}\hline
%    x & $\alpha$-code & $\beta$-code & $\beta'$-code & $\gamma'$-code & $\gamma$-code & $\delta$-code & $\omega$-code & $\omega'$-code \\\hline
%    0  & -         & $0$      & -       & $\blue1$      & $\blue1$      & $\blue1$      & & \\
%    1  & $1$       & $1$      & $\varepsilon$ & & & & & \\
%    2  & $01$      & $10$     & $0$     & $\blue01\red0$     & $0$     & $0$     & & & \\
%    3  & $001$     & $11$     & $1$     & $\blue01\red1$     & $1$     & $1$     & & \\
%    4  & $0^31$    & $100$    & $00$    & $\blue001\red00$    & $00$    & $00$    & & \\
%    5  & $0^41$    & $101$    & $01$    & $\blue001\red01$    & $01$    & $01$    & & \\
%    6  & $0^51$    & $110$    & $10$    & $\blue001\red10$    & $10$    & $10$    & & \\
%    7  & $0^61$    & $110$    & $11$    & $\blue001\red10$    & $10$    & $10$    & & \\
%    8  & $0^71$    & $1000$   & $000$   & $\blue0001\red000$   & $000$   & $000$   & & \\
%    9  & $0^81$    & $1001$   & $001$   & $\blue0001\red001$   & $001$   & $001$   & & \\
%    10 & $0^91$    & $1010$   & $010$   & $\blue0001\red010$   & $010$   & $010$   & & \\
%    11 & $0^{10}1$ & $1011$   & $011$   & $\blue0001\red011$   & $011$   & $011$   & & \\
%    12 & $0^{11}1$ & $1100$   & $100$   & $\blue0001\red100$   & $100$   & $100$   & & \\
%    13 & $0^{12}1$ & $1101$   & $101$   & $\blue0001\red101$   & $101$   & $101$   & & \\
%    14 & $0^{13}1$ & $1110$   & $110$   & $\blue0001\red110$   & $110$   & $110$   & & \\
%    15 & $0^{14}1$ & $1111$   & $111$   & $\blue0001\red111$   & $111$   & $111$   & & \\
%    16 & $0^{15}1$ & $10000$  & $0000$  & $\blue00001\red0000$  & $0000$  & $0000$  & & \\
%    23 & $0^{22}1$ & $10111$  & $0111$  & $\blue00001\red0111$  & $0111$  & $0111$  & & \\
%    24 & $0^{23}1$ & $11000$  & $1000$  & $\blue00001\red1000$  & $1000$  & $1000$  & & \\
%    31 & $0^{30}1$ & $11111$  & $1111$  & $\blue00001\red1111$  & $1111$  & $1111$  & & \\
%    32 & $0^{31}1$ & $100000$ & $00000$ & $\blue000001\red00000$ & $00000$ & $00000$ & & \\\hline
%\end{tabular}
%\\
%\caption{Examples of coding of integers}
%\end{table*}
